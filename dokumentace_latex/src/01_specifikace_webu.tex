\chapter{Specifikace webu}

\section{Představení řešeného webu}

\subsection{Název projektu}
\textbf{Rehab from Vibe Coding}

\subsection{Popis zařízení}
Jedná se o fiktivní rehabilitační centrum specializující se na léčbu vývojářů
závislých na AI asistovaném programování. Centrum nabízí terapeutické programy
zaměřené na obnovu fundamentálních programovacích dovedností, sémantického HTML,
CSS bez frameworků, přístupnosti a bezpečnosti kódu.

\subsection{Koncept webu}
Web satiricky reflektuje současný trend nadměrného spoléhání se na AI nástroje
při vývoji softwaru. Nabízí humorný pohled na problémy vývojářů, kteří ztratili
schopnost psát kvalitní kód bez asistence AI.

\section{Struktura a popis stránek}

\subsection{Úvodní stránka (index.html)}
\begin{itemize}
    \item Hero sekce s výrazným nadpisem a popisem služeb
    \item Sekce ``About Our Mission'' s představením poslání centra
    \item Tabulka příznaků AI závislosti s doporučenými programy
    \item Glosář odborných pojmů (Vibe Coding, Div Soup, atd.)
    \item Sekce s přehledem léčebných programů
    \item Call-to-action sekce pro rezervaci konzultace
    \item Testimonials (úspěšné příběhy klientů)
    \item Plnohodnotná patička s kontakty a informacemi o týmu
\end{itemize}

\subsection{Seznam terapeutů (pages/therapists/list.html)}
\begin{itemize}
    \item Přehled 8 terapeutů v grid layoutu
    \item Každý terapeut má fotku, jméno, specializaci a stručný popis
    \item Odkazy na detailní profily jednotlivých terapeutů
    \item Vyhledávací pole pro konzistenci napříč stránkami
\end{itemize}

\subsection{Detailní profily terapeutů (8 samostatných stránek)}
\begin{enumerate}
    \item \textbf{emma-html.html} -- Dr. Emma HTML (Semantic Markup Specialist)
    \item \textbf{marcus-grid.html} -- Prof. Marcus Grid (Layout Architecture Expert)
    \item \textbf{aria-accessible.html} -- Dr. Aria Accessible (Web Accessibility Advocate)
    \item \textbf{flex-box.html} -- Dr. Flex Box (Component Layout Specialist)
    \item \textbf{val-root.html} -- Specialist Val Root (CSS Variables \& Architecture)
    \item \textbf{sam-security.html} -- Dr. Sam Security (Code Security Consultant)
    \item \textbf{petra-perf.html} -- Prof. Petra Perf (Performance Optimization)
    \item \textbf{min-code.html} -- Dr. Min Code (Mindful Coding Coach)
\end{enumerate}

Každý profil obsahuje:
\begin{itemize}
    \item Portrét a základní informace
    \item Detailní biografii
    \item Seznam specializací a odborností
    \item Publikace a články
    \item Vzdělání a certifikace
    \item Terapeutický přístup
    \item Recenze od klientů
    \item Call-to-action pro rezervaci sezení
\end{itemize}

\subsection{Vyhledávání}
Vyhledávací pole je přítomno na všech stránkách, připraveno pro budoucí
implementaci funkčního vyhledávání.

\section{Cílové skupiny uživatelů}

\subsection{Primární cílové skupiny}
\begin{itemize}
    \item Junior vývojáři závislí na AI nástrojích (Copilot, ChatGPT)
    \item Frontend vývojáři s nedostatečnými znalostmi HTML/CSS
    \item Vývojáři používající pouze CSS frameworky (Bootstrap, Tailwind)
    \item Senior vývojáři, kteří ztratili přehled o základech
    \item Týmoví lídři hledající vzdělávací programy pro své týmy
    \item HR manažeři zodpovědní za rozvoj vývojářů
\end{itemize}

\subsection{Sekundární cílové skupiny}
\begin{itemize}
    \item Studenti informatiky potřebující doplnit znalosti
    \item Kariérní přesmyčkáři vstupující do webového vývoje
    \item Freelanceři chtějící zlepšit kvalitu svého kódu
\end{itemize}

\section{Cíl webu}

\subsection{Hlavní cíle}
\begin{itemize}
    \item Představit rehabilitační centrum a jeho služby
    \item Informovat o problematice AI závislosti ve vývoji softwaru
    \item Nabídnout edukační programy zaměřené na fundamentální dovednosti
    \item Umožnit návštěvníkům najít vhodného terapeuta pro své potřeby
    \item Poskytovat vzdělávací obsah o sémantickém HTML, CSS a přístupnosti
\end{itemize}

\subsection{Funkční cíle}
\begin{itemize}
    \item Prezentace 8 odborníků s jejich specializacemi
    \item Přehledná navigace mezi sekcemi webu
    \item Responzivní design pro všechna zařízení
    \item Vysoká úroveň přístupnosti (WCAG 2.0 AA)
    \item Možnost budoucího rozšíření o rezervační systém
\end{itemize}

\section{Požadavky na implementaci}

\subsection{HTML požadavky}
\begin{itemize}
    \item HTML5 s korektní strukturou
    \item Sémantické elementy (article, section, header, footer, aside, nav, main)
    \item Správné použití nadpisů h1, h2, h3
    \item Tabulka pro strukturovaná data
    \item Formulářové elementy pro vyhledávání
    \item Relativní adresování všech odkazů a zdrojů
    \item Meta tagy pro SEO a responzivitu
\end{itemize}

\subsection{CSS požadavky}
\begin{itemize}
    \item Vlastní CSS bez použití frameworků
    \item CSS Custom Properties (proměnné) pro barvy, spacing, typography
    \item CSS Grid layout s minimálně 8 elementy
    \item Flexbox pro layouty komponent
    \item Responzivní design s Mobile-first přístupem
    \item Media queries pro tablet (768px+) a desktop (1024px+)
\end{itemize}

\subsection{Přístupnost}
\begin{itemize}
    \item WCAG 2.0 Level AA compliance
    \item Skip links pro klávesnicovou navigaci
    \item ARIA labels a roles kde potřebné
    \item Sémantické HTML pro screen readery
    \item Dostatečný kontrast barev
    \item Focus states pro interaktivní elementy
\end{itemize}

\subsection{Chování stránek}
\begin{itemize}
    \item Responzivní layout se přizpůsobí šířce obrazovky
    \item Grid layout se mění: 1 sloupec (mobil) $\rightarrow$ 2 sloupce (tablet)
          $\rightarrow$ 3-4 sloupce (desktop)
    \item Navigace se transformuje z vertikálního menu na horizontální
    \item Hover efekty na kartách terapeutů a tlačítkách
    \item Smooth transitions pro interaktivní elementy
    \item Print styles pro tisk dokumentů
\end{itemize}

\section{Návrh technologií}

\subsection{Frontend}
\begin{description}
    \item[HTML5] Nejnovější standard pro strukturu dokumentu
    \item[CSS3] Moderní styling s Grid, Flexbox, Custom Properties
    \item[JavaScript] Žádný -- zaměření na čisté HTML/CSS řešení
    \item[Přístup] Responzivní Mobile-first
\end{description}

\subsection{Zdůvodnění volby technologií}
\begin{itemize}
    \item HTML5 poskytuje bohatou sadu sémantických elementů
    \item CSS Grid umožňuje vytvářet komplexní layouty bez frameworků
    \item CSS Custom Properties zajišťují konzistenci a snadnou údržbu
    \item Mobile-first přístup odpovídá současným trendům v použití zařízení
    \item Absence JavaScriptu demonstruje sílu moderního CSS
\end{itemize}

\subsection{Backend (budoucí rozšíření)}
\begin{description}
    \item[Node.js s Express.js] Pro API a server-side logiku
    \item[PostgreSQL] Relační databáze pro uživatele a rezervace
    \item[REST API] Pro komunikaci mezi frontendem a backendem
\end{description}

\section{Nasazení}

\subsection{Hosting}
\begin{description}
    \item[Platforma] GitHub Pages
    \item[URL] \url{https://katyabiser.github.io/Rehab-from-Vibe-Coding/}
    \item[Výhody] Zdarma, jednoduchý deployment, automatické HTTPS
\end{description}

\subsection{Budoucí rozšíření hostingu}
\begin{itemize}
    \item Vercel nebo Netlify -- pro statické stránky s lepšími funkcemi
    \item Heroku nebo DigitalOcean -- při implementaci backendu
    \item Cloudflare CDN -- pro rychlejší načítání globálně
\end{itemize}

\subsection{Údržba}
\begin{itemize}
    \item Verzování pomocí Git/GitHub
    \item Automatický deployment při push do main branch
    \item Pravidelné aktualizace obsahu
    \item Kontrola funkčnosti odkazů
    \item Monitorování accessibility compliance
\end{itemize}

\subsection{Analytika (budoucí implementace)}
\begin{itemize}
    \item Google Analytics -- sledování návštěvnosti
    \item Hotjar -- heatmapy a analýza chování uživatelů
    \item Lighthouse CI -- automatické testování výkonu a přístupnosti
\end{itemize}

\subsection{Marketing (budoucí plány)}
\begin{itemize}
    \item SEO optimalizace (meta tagy, strukturovaná data)
    \item Social media integrace
    \item Blog s edukačním obsahem o web developmentu
    \item Newsletter pro pravidelnou komunikaci s uživateli
\end{itemize}

\subsection{Kontinuální UX}
\begin{itemize}
    \item A/B testování různých verzí call-to-action
    \item Uživatelské testování s reálnými vývojáři
    \item Iterativní vylepšování na základě feedbacku
    \item Accessibility audity s asistivními technologiemi
    \item Performance monitoring a optimalizace
\end{itemize}
