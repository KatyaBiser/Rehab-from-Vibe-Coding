\chapter{Úvod}

\section{Představení projektu}

\begin{itemize}
  \item Web fiktivního centra \textbf{Rehab from Vibe Coding} satiricky reflektuje závislost na AI při vývoji.
  \item Humor + edukace: ukazuje, proč se vyplatí psát sémantický a udržitelný kód místo slepého spoléhání na AI.
  \item Terapeutické „programy“ symbolizují návrat k fundamentálním dovednostem (HTML, čisté CSS, přístupnost, bezpečnost).
  \item Poselství: AI má pomáhat, ne nahrazovat; lidské znalosti a kontrola zůstávají klíčové.
\end{itemize}

\section{Struktura webu}

\begin{itemize}
  \item Homepage s misí centra, službami a tabulkou příznaků AI závislosti.
  \item Seznam devíti terapeutů v responzivním grid layoutu.
  \item Devět detailních profilů (bio, expertíza, recenze, CTA na rezervaci).
  \item Technické detaily struktury jsou rozvedeny v kapitole \ref{chap:pozadavky}.
\end{itemize}


\section{Cílová skupina a účel}

\begin{itemize}
  \item Primární publikum: vývojáři závislí na AI (``vibe coding''), kteří ztrácejí fundamentální dovednosti.
  \item Sekundární publikum: týmoví lídři/senioři pro osvětu, studenti informatiky pro dobré návyky.
  \item Edukační účel: připomenout hodnotu sémantického HTML, přístupnosti, bezpečnosti a čistého CSS bez frameworků.
  \item Humorný účel: satira na nadměrné spoléhání na AI, upozornění na rizika (zranitelnosti, nekvalitní kód).
\end{itemize}

\section{Technologický přístup}

\begin{itemize}
  \item Čisté webové technologie bez frameworků: sémantické HTML + CSS (Grid, Flexbox, Custom Properties).
  \item Interaktivita čistě v CSS: checkbox pro navigaci a vyhledávání, animace a focus stavy.
  \item Mobile-first, plně responzivní; splnění WCAG 2.0 Level AA.
  \item Nasazení na GitHub Pages: \url{https://katyabiser.github.io/Rehab-from-Vibe-Coding/}.
  \item Technické detaily a volby technologií jsou rozvedeny v kapitole \ref{chap:pozadavky}.
\end{itemize}
