\chapter{Úvod}

\section{Představení projektu}

Tento projekt představuje webové stránky fiktivního rehabilitačního centra s názvem \textbf{Rehab from Vibe Coding}. Web satiricky reflektuje současný trend nadměrného spoléhání se na nástroje umělé inteligence při vývoji softwaru. S humorným nadhledem poukazuje na problémy vývojářů, kteří postupně ztrácejí schopnost psát kvalitní, sémantický a udržitelný kód bez asistence AI.

Cílem projektu je nejen pobavit, ale také vzdělávat. Prezentuje rehabilitační centrum, které nabízí terapeutické programy zaměřené na obnovu fundamentálních programovacích dovedností -- od psaní sémantického HTML a stylování pomocí čistého CSS bez frameworků, až po zajištění přístupnosti podle standardu WCAG 2.0 AA a bezpečnosti kódu. Tímto způsobem projekt demonstruje a obhajuje důležitost základních znalostí, které jsou v moderním vývoji často opomíjeny ve prospěch rychlých AI-generovaných řešení.

\section{Struktura webu}

Web se skládá ze tří hlavních částí: úvodní stránky představující poslání centra a přehled služeb, seznamu devíti terapeutů v responzivním grid layoutu a detailních profilů jednotlivých odborníků s biografiemi, expertízou a recenzemi. Detailní technické specifikace struktury a obsahu jednotlivých stránek jsou popsány v kapitole \ref{chap:pozadavky}.

\section{Tým terapeutů}

Rehabilitační centrum zaměstnává devět specializovaných terapeutů, z nichž každý se zaměřuje na specifický aspekt fundamentálních programovacích dovedností:

\textbf{Dr. Emma Hypertext} (Semantic Markup Specialist) se specializuje na léčbu div-soup syndromu a pomáhá vývojářům objevit sémantické HTML elementy. Její biografie popisuje traumatický zážitek z roku 2005, kdy viděla web postavený výhradně z vnořených div tagů, což ji inspirovalo k založení Semantic Recovery Institute.

\textbf{Prof. Marcus Grid} (Layout Architecture Expert) je průkopníkem moderních CSS layout technik, zejména CSS Grid. Pomáhá vývojářům uniknout ze závislosti na frameworkech a naučit se vytvářet vlastní responzivní layouty od základů.

\textbf{Dr. Aria Accessible} (Web Accessibility Advocate) je certifikovaný specialista na standard WCAG 2.0 AA. Její mise je učinit web přístupný pro všechny uživatele bez ohledu na jejich schopnosti a technologie, které používají.

\textbf{Dr. Flexbox Layout} (Component Layout Specialist) je mistrem flexibilních box layoutů a responzivního designu. Specializuje se na komponenty a ``dělá zarovnávání opět jednoduchým''.

\textbf{Specialist Val Root} (CSS Variables \& Architecture) vyučuje vývojáře psát udržitelné a škálovatelné CSS pomocí custom properties a moderní architektury stylů. Zaměřuje se na organizaci kódu a best practices.

\textbf{Dr. Samuel Security} (Code Security Consultant) se specializuje na identifikaci a prevenci bezpečnostních zranitelností v AI-generovaném kódu. Je certifikován OWASP a pomáhá vývojářům pochopit bezpečnostní dopady kódu, který nekriticky přijímají od AI asistentů.

\textbf{Prof. Petra Performance} (Performance Optimization) učí vývojáře psát štíhlý a rychlý kód bez závislosti na těžkých frameworkcích a zbytečných závislostech. Zaměřuje se na optimalizaci výkonu a best practices pro rychlé načítání webových stránek.

\textbf{Dr. Mindful Coding} (Mindful Coding Coach) propaguje vědomé a záměrné programování. Pomáhá vývojářům pochopit ``proč'' za jejich rozhodnutími a učí je přemýšlet o každém řádku kódu, který píší, namísto mechanického kopírování AI návrhů.

\textbf{Dr. Git Commit-Log} (Version Control Counselor) je nejnovějším přírůstkem týmu. Specializuje se na řešení ``emocionálních merge konfliktů'' a léčbu chronické commit anxiety. Pomáhá vývojářům, kteří trpí Main Branch Fear nebo Stashing Hoarding Disorder, a učí je zdravé návyky verzovacího systému.

\section{Cílová skupina a účel}

Projekt je určen pro všechny vývojáře bez ohledu na úroveň zkušeností, kteří se stali závislými na AI asistovaném programování (tzv. ``vibe coding'') a ztratili kontakt s fundamentálními dovednostmi. Primární cílovou skupinu tvoří vývojáři, kteří nekriticky přijímají AI návrhy bez porozumění kódu, spoléhají se výhradně na frameworky a generativní nástroje, nebo ztratili schopnost psát čistý HTML/CSS bez asistence. Sekundární skupinu představují týmoví lídři a senior vývojáři hledající vzdělávací materiál o rizicích nadměrné AI závislosti, stejně jako studenti informatiky, kteří si chtějí osvojit správné návyky a best practices ještě před vznikem závislosti na AI nástrojích.

Web má dvojí účel: vzdělávací a humorný. Na jedné straně poskytuje skutečné informace o důležitosti sémantického HTML, přístupnosti, bezpečnosti a čistého CSS bez frameworků. Na druhé straně používá satiru a humor k tomu, aby upozornil na reálný problém nadměrné závislosti na AI nástrojích, který může vést ke ztrátě fundamentálních dovedností, bezpečnostním zranitelnostem a nekvalitnímu kódu.

\section{Technologický přístup}

Projekt je záměrně implementován pomocí čistých webových technologií bez použití frameworků -- HTML5 se sémantickými elementy, CSS3 s moderními technikami (Grid, Flexbox, Custom Properties) a vanilla JavaScript. Tento přístup není pouze koncepční, ale slouží jako manifest hodnot, které centrum propaguje: fundamentální dovednosti před závislostí na nástrojích.

Web je plně responzivní s Mobile-first přístupem a splňuje standard WCAG 2.0 Level AA pro přístupnost. Projekt je nasazen na GitHub Pages a dostupný na adrese \url{https://katyabiser.github.io/Rehab-from-Vibe-Coding/}. Detailní technické specifikace, zdůvodnění výběru technologií a implementační detaily jsou popsány v kapitole \ref{chap:pozadavky}.
