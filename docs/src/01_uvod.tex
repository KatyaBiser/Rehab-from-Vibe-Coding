\chapter{Úvod}

\section{Představení projektu}

Tento projekt představuje webové stránky fiktivního rehabilitačního centra s názvem \textbf{Rehab from Vibe Coding}. Web satiricky reflektuje současný trend nadměrného spoléhání se na nástroje umělé inteligence při vývoji softwaru. S humorným nadhledem poukazuje na problémy vývojářů, kteří postupně ztrácejí schopnost psát kvalitní, sémantický a udržitelný kód bez asistence AI.

Cílem projektu je nejen pobavit, ale také vzdělávat. Prezentuje rehabilitační centrum, které nabízí terapeutické programy zaměřené na obnovu fundamentálních programovacích dovedností -- od psaní sémantického HTML a stylování pomocí čistého CSS bez frameworků, až po zajištění přístupnosti podle standardu WCAG 2.0 AA a bezpečnosti kódu. Tímto způsobem projekt demonstruje a obhajuje důležitost základních znalostí, které jsou v moderním vývoji často opomíjeny ve prospěch rychlých AI-generovaných řešení.

\section{Struktura webu}

Web se skládá ze tří hlavních částí: úvodní stránky (index.html), seznamu terapeutů (pages/therapists/list.html) a devíti detailních profilů jednotlivých odborníků.

\subsection{Úvodní stránka}

Úvodní stránka obsahuje hero sekci s výrazným nadpisem \textit{"Tired of AI Writing Code for You?"} a motivačním popisem služeb. Následuje sekce \textit{"About Our Mission"}, která představuje poslání centra a vysvětluje problematiku AI závislosti ve vývoji. Klíčovou součástí je interaktivní tabulka příznaků AI závislosti, která uvádí osm běžných symptomů (například ``Accepting all AI suggestions without reading them'' nebo ``Can't write CSS without Tailwind/Bootstrap''), jejich závažnost, doporučený léčebný program a délku léčby.

Dále stránka obsahuje glosář odborných pojmů, který definuje termíny jako \textit{Vibe Coding} (přijímání AI návrhů bez porozumění), \textit{Div Soup} (nadužívání nesémantických elementů), \textit{Semantic Recovery} (proces učení sémantického HTML) a \textit{Framework Dependency} (nadměrná závislost na CSS frameworkech). Sekce s přehledem léčebných programů nabízí úvod do terapeutických služeb a call-to-action sekce vyzývá k rezervaci konzultace.

V dolní části stránky jsou umístěny testimonials -- citace úspěšných klientů, které humorně dokumentují cestu od AI závislosti k plné kontrole nad vlastním kódem. Například Sarah K. (Frontend Developer) popisuje, jak ``přestala klikat 'accept all suggestions' bez čtení'' nebo James L. (Senior Engineer) vysvětluje, jak program ``zachránil jeho reputaci po neúspěšných bezpečnostních auditech''. Stránka je zakončena plnohodnotnou patičkou s kontakty, odkazy na programy, informacemi o týmu a upozorněním, že se jedná o studentský projekt.

\subsection{Seznam terapeutů}

Stránka se seznamem terapeutů prezentuje devět odborníků v responzivním grid layoutu. Každý terapeut je představen kartou obsahující portrét, jméno, specializaci, stručný popis expertízy a odkaz na detailní profil. Vybraní terapeuti mají označení ``Lead Therapist'' s informací ``Currently accepting limited clients'', což vytváří autentický dojem exkluzivity služeb. V horní části stránky je umístěn úvodní text představující tým a jejich kombinované zkušenosti v oblasti léčby AI závislosti.

\subsection{Detailní profily terapeutů}

Každý z devíti terapeutů má vlastní dedikovanou stránku s komplexním profilem. Stránky používají breadcrumb navigaci pro lepší orientaci uživatele (Home $\rightarrow$ Our Therapists $\rightarrow$ Jméno terapeuta). Profil začíná velkým portrétem, jménem a podtitulem specializace, následuje detailní biografie popisující humornou backstory terapeuta a jeho motivaci pomáhat vývojářům. Dále jsou uvedeny oblasti expertízy (5-8 specializací), seznam vybraných publikací s fiktivními tituly článků a knihy, vzdělání a certifikace, popis terapeutického přístupu a recenze od bývalých klientů. Každý profil končí call-to-action sekcí s výzvou k rezervaci sezení.

\section{Tým terapeutů}

Rehabilitační centrum zaměstnává devět specializovaných terapeutů, z nichž každý se zaměřuje na specifický aspekt fundamentálních programovacích dovedností:

\textbf{Dr. Emma Hypertext} (Semantic Markup Specialist) se specializuje na léčbu div-soup syndromu a pomáhá vývojářům objevit sémantické HTML elementy. Její biografie popisuje traumatický zážitek z roku 2005, kdy viděla web postavený výhradně z vnořených div tagů, což ji inspirovalo k založení Semantic Recovery Institute.

\textbf{Prof. Marcus Grid} (Layout Architecture Expert) je průkopníkem moderních CSS layout technik, zejména CSS Grid. Pomáhá vývojářům uniknout ze závislosti na frameworkech a naučit se vytvářet vlastní responzivní layouty od základů.

\textbf{Dr. Aria Accessible} (Web Accessibility Advocate) je certifikovaný specialista na standard WCAG 2.0 AA. Její mise je učinit web přístupný pro všechny uživatele bez ohledu na jejich schopnosti a technologie, které používají.

\textbf{Dr. Flexbox Layout} (Component Layout Specialist) je mistrem flexibilních box layoutů a responzivního designu. Specializuje se na komponenty a ``dělá zarovnávání opět jednoduchým''.

\textbf{Specialist Val Root} (CSS Variables \& Architecture) vyučuje vývojáře psát udržitelné a škálovatelné CSS pomocí custom properties a moderní architektury stylů. Zaměřuje se na organizaci kódu a best practices.

\textbf{Dr. Samuel Security} (Code Security Consultant) se specializuje na identifikaci a prevenci bezpečnostních zranitelností v AI-generovaném kódu. Je certifikován OWASP a pomáhá vývojářům pochopit bezpečnostní dopady kódu, který nekriticky přijímají od AI asistentů.

\textbf{Prof. Petra Performance} (Performance Optimization) učí vývojáře psát štíhlý a rychlý kód bez závislosti na těžkých frameworkcích a zbytečných závislostech. Zaměřuje se na optimalizaci výkonu a best practices pro rychlé načítání webových stránek.

\textbf{Dr. Mindful Coding} (Mindful Coding Coach) propaguje vědomé a záměrné programování. Pomáhá vývojářům pochopit ``proč'' za jejich rozhodnutími a učí je přemýšlet o každém řádku kódu, který píší, namísto mechanického kopírování AI návrhů.

\textbf{Dr. Git Commit-Log} (Version Control Counselor) je nejnovějším přírůstkem týmu. Specializuje se na řešení ``emocionálních merge konfliktů'' a léčbu chronické commit anxiety. Pomáhá vývojářům, kteří trpí Main Branch Fear nebo Stashing Hoarding Disorder, a učí je zdravé návyky verzovacího systému.

\section{Cílová skupina a účel}

Projekt je primárně určen pro junior vývojáře, kteří se učí programovat v éře všudypřítomných AI asistentů a potřebují pochopit fundamenty před použitím pokročilých nástrojů. Dále cílí na frontend developery s nedostatečnými znalostmi HTML a CSS, kteří se spoléhají výhradně na frameworky a generativní nástroje. Třetí skupinou jsou týmoví lídři a senior vývojáři, kteří hledají vzdělávací materiál pro své juniory, a poslední kategorii tvoří studenti informatiky, kteří si osvojují best practices v oblasti webového vývoje.

Web má dvojí účel: vzdělávací a humorný. Na jedné straně poskytuje skutečné informace o důležitosti sémantického HTML, přístupnosti, bezpečnosti a čistého CSS bez frameworků. Na druhé straně používá satiru a humor k tomu, aby upozornil na reálný problém nadměrné závislosti na AI nástrojích, který může vést ke ztrátě fundamentálních dovedností, bezpečnostním zranitelnostem a nekvalitnímu kódu.

\section{Technologický přístup}

Projekt je záměrně implementován pomocí čistých webových technologií bez použití moderních frameworků. Využívá HTML5 s důrazem na sémantické elementy (article, section, header, footer, aside, nav, main), CSS3 s moderními technikami (Grid, Flexbox, Custom Properties) a vanilla JavaScript pro interaktivitu. Tento přístup není pouze koncepční -- samotná implementace webu slouží jako manifest a praktická ukázka dovedností, které centrum ``léčí''.

Design je plně responzivní s Mobile-first přístupem, který zajišťuje optimální zobrazení na všech zařízeních od mobilních telefonů po desktopy. Navigace se transformuje z hamburger menu na mobilech na horizontální menu na větších obrazovkách, grid layouty se přizpůsobují od jednoho sloupce (mobil) přes dva sloupce (tablet) až po tři nebo čtyři sloupce (desktop). JavaScript (navigation.js) zajišťuje plynulé přepínání mobilního menu, správu ARIA atributů pro přístupnost a funkcionalitu vyhledávacího pole s dropdown animací.

Web klade velký důraz na přístupnost podle standardu WCAG 2.0 Level AA. Obsahuje skip links pro snadnou klávesnicovou navigaci, ARIA labels a roles kde jsou potřebné, sémantické HTML pro screen readery, dostatečný kontrast barev a viditelné focus states pro všechny interaktivní elementy. Projekt je nasazen na GitHub Pages a je veřejně dostupný na adrese \url{https://katyabiser.github.io/Rehab-from-Vibe-Coding/}.
