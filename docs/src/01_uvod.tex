\chapter{Úvod}

\section{Představení projektu}

Tento projekt představuje webové stránky fiktivního rehabilitačního centra s názvem \textbf{Rehab from Vibe Coding}. Web satiricky reflektuje současný trend nadměrného spoléhání se na nástroje umělé inteligence při vývoji softwaru. S humorným nadhledem poukazuje na problémy vývojářů, kteří postupně ztrácejí schopnost psát kvalitní, sémantický a udržitelný kód bez asistence AI.

Cílem projektu je nejen pobavit, ale také vzdělávat. Prezentuje rehabilitační centrum, které nabízí terapeutické programy zaměřené na obnovu fundamentálních programovacích dovedností -- od psaní sémantického HTML a stylování pomocí čistého CSS bez frameworků, až po zajištění přístupnosti podle standardu WCAG 2.0 AA a bezpečnosti kódu. Tímto způsobem projekt demonstruje a obhajuje důležitost základních znalostí, které jsou v moderním vývoji často opomíjeny ve prospěch rychlých AI-generovaných řešení.

\section{Struktura webu}

Web se skládá ze tří hlavních částí: úvodní stránky představující poslání centra a přehled služeb, seznamu devíti terapeutů v responzivním grid layoutu a detailních profilů jednotlivých odborníků s biografiemi, expertízou a recenzemi. Detailní technické specifikace struktury a obsahu jednotlivých stránek jsou popsány v kapitole \ref{chap:pozadavky}.


\section{Cílová skupina a účel}

Projekt je určen pro všechny vývojáře bez ohledu na úroveň zkušeností, kteří se stali závislými na AI asistovaném programování (tzv. ``vibe coding'') a ztratili kontakt s fundamentálními dovednostmi. Primární cílovou skupinu tvoří vývojáři, kteří nekriticky přijímají AI návrhy bez porozumění kódu, spoléhají se výhradně na frameworky a generativní nástroje, nebo ztratili schopnost psát čistý HTML/CSS bez asistence. Sekundární skupinu představují týmoví lídři a senior vývojáři hledající vzdělávací materiál o rizicích nadměrné AI závislosti, stejně jako studenti informatiky, kteří si chtějí osvojit správné návyky a best practices ještě před vznikem závislosti na AI nástrojích.

Web má dvojí účel: vzdělávací a humorný. Na jedné straně poskytuje skutečné informace o důležitosti sémantického HTML, přístupnosti, bezpečnosti a čistého CSS bez frameworků. Na druhé straně používá satiru a humor k tomu, aby upozornil na reálný problém nadměrné závislosti na AI nástrojích, který může vést ke ztrátě fundamentálních dovedností, bezpečnostním zranitelnostem a nekvalitnímu kódu.

\section{Technologický přístup}

Projekt je záměrně implementován pomocí čistých webových technologií bez použití frameworků -- HTML5 se sémantickými elementy a CSS3 s moderními technikami (Grid, Flexbox, Custom Properties). Interaktivita je realizována pomocí checkbox elementů a selektorů pro navigaci a vyhledávání. Tento přístup není pouze koncepční, ale slouží jako manifest hodnot, které centrum propaguje: fundamentální dovednosti před závislostí na nástrojích.

Web je plně responzivní s Mobile-first přístupem a splňuje standard WCAG 2.0 Level AA pro přístupnost. Projekt je nasazen na GitHub Pages a dostupný na adrese \url{https://katyabiser.github.io/Rehab-from-Vibe-Coding/}. Detailní technické specifikace, zdůvodnění výběru technologií a implementační detaily jsou popsány v kapitole \ref{chap:pozadavky}.
