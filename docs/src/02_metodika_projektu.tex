\chapter{Metodika projektu}

\section{Software, nástroje a IDE}

\subsection{Vývojové prostředí}
\begin{description}
    \item[Visual Studio Code] Hlavní IDE
    \begin{itemize}
        \item Rozšíření: Live Server pro lokální preview
        \item Rozšíření: HTML CSS Support pro autocomplete
        \item Rozšíření: Prettier pro formátování kódu
        \item Rozšíření: axe Accessibility Linter
    \end{itemize}
\end{description}

\subsection{Design a prototypování}
\begin{itemize}
    \item Figma -- wireframing a mockupy (používala Persona B)
    \item Papír a tužka -- rychlé skici layoutu
\end{itemize}

\subsection{Správa verzí}
\begin{itemize}
    \item Git -- verzování kódu
    \item GitHub -- hosting repository a collaboration
    \item GitHub Desktop -- GUI pro Git (pro méně technické členy týmu)
\end{itemize}

\subsection{Generování obsahu}
\begin{itemize}
    \item GPT 5.1 -- generování textů, biografií, recenzí pro web
    \item Claude Code (Sonnet 4.5) -- asistence při psaní kódu
    \item Google Images -- vyhledávání fotografií lékařů a rehabilitačních center
\end{itemize}

\subsection{Testování a validace}
\begin{itemize}
    \item W3C HTML Validator -- kontrola validity HTML
    \item W3C CSS Validator -- kontrola validity CSS
    \item WAVE Web Accessibility Evaluation Tool
    \item axe DevTools -- accessibility testing v prohlížeči
    \item Lighthouse -- performance a accessibility audit
    \item BrowserStack -- testování v různých prohlížečích
\end{itemize}

\subsection{Grafické úpravy}
\begin{itemize}
    \item Adobe Photoshop -- úprava AI generovaných obrázků
    \item TinyPNG -- komprese obrázků pro web
\end{itemize}

\subsection{Prohlížeče pro testování}
\begin{itemize}
    \item Google Chrome + DevTools
    \item Mozilla Firefox + Developer Tools
    \item Safari (macOS)
    \item Microsoft Edge
\end{itemize}

\section{Využití AI nástrojů při tvorbě projektu}

\subsection{Generování textového obsahu}

\subsubsection{Použité nástroje}
\begin{itemize}
    \item GPT 5.1
\end{itemize}

\subsubsection{Jak byly využity}
\begin{itemize}
    \item Generování humorných biografií terapeutů
    \item Vytváření konzistentního tone of voice
    \item Psaní recenzí od fiktivních klientů
    \item Generování tabulky příznaků AI závislosti
    \item Tvorba glosáře odborných pojmů
    \item Vytváření textového obsahu pro všechny stránky webu
\end{itemize}

\subsubsection{Příklady promptů}
\begin{lstlisting}[style=htmlstyle, caption={Prompt pro tvorbu profilu terapeuta}]
Vytvor humorny profil fiktivniho terapeuta Dr. Emma HTML,
ktera se specializuje na lecbu vyvojaru zavislych na pouzivani
<div> elementu misto semantickeho HTML. Zahrn biografii,
specializace a recenze od klientu.
Ton: satiricky, ale empaticky.
\end{lstlisting}

\subsection{Vizuální obsah}

\subsubsection{Použité nástroje}
\begin{itemize}
    \item Google Images -- vyhledávání fotografií
    \item Unsplash -- stock fotografie lékařů a zdravotnických zařízení
    \item Adobe Photoshop -- úprava a optimalizace obrázků
\end{itemize}

\subsubsection{Jak byly využity}
\begin{itemize}
    \item Vyhledání vhodných fotografií lékařů pro profily terapeutů
    \item Výběr obrázků rehabilitačních center a zdravotnických zařízení
    \item Úprava velikosti a optimalizace obrázků pro web
    \item Zajištění konzistence vizuálního stylu napříč stránkami
\end{itemize}

\subsection{Asistence při psaní kódu}

\subsubsection{Použité nástroje}
\begin{itemize}
    \item Claude Code (Sonnet 4.5) -- asistent pro psaní HTML a CSS kódu
    \item GitHub Copilot -- code reviewer v pull requests
\end{itemize}

\subsubsection{Jak byly využity}
\begin{itemize}
    \item \textbf{Claude Code:} Řešení CSS Grid problémů s layout
    \item \textbf{Claude Code:} Kontrola accessibility best practices
    \item \textbf{Claude Code:} Generování CSS custom properties struktur
    \item \textbf{Claude Code:} Návody na responzivní design patterns
    \item \textbf{Claude Code:} Psaní a refaktoring HTML/CSS kódu
    \item \textbf{GitHub Copilot:} Automatická code review v pull requests
    \item \textbf{GitHub Copilot:} Kontrola a validace změn před merge
    \item \textbf{GitHub Copilot:} Navrhování vylepšení kódu v PR komentářích
\end{itemize}

\textbf{POZOR:} Většina HTML/CSS byla napsána ručně bez AI asistence,
aby projekt skutečně demonstroval fundamentální dovednosti.

\section{Kritické zhodnocení AI nástrojů}

\subsection{Přínosy}

\subsubsection{Content generation -- GPT 5.1 (9/10)}
\begin{itemize}
    \item[+] Rychlé vytváření konzistentního obsahu
    \item[+] Kreativní nápady pro humorné texty
    \item[+] Úspora času při psaní opakujícího se obsahu
    \item[+] Schopnost generovat různé varianty textu
    \item[+] Vysoká kvalita gramatiky a stylu
\end{itemize}

\subsubsection{Image sourcing -- Google Images / Unsplash (8/10)}
\begin{itemize}
    \item[+] Široký výběr profesionálních fotografií
    \item[+] Realistické portréty skutečných lidí
    \item[+] Vysoká kvalita a rozlišení obrázků
    \item[+] Snadné vyhledávání podle klíčových slov
    \item[-] Časově náročnější než AI generování
    \item[-] Nutnost kontroly licencí a autorských práv
\end{itemize}

\subsubsection{Code assistance -- Claude Code Sonnet 4.5 (8/10)}
\begin{itemize}
    \item[+] Velmi rychlé řešení specifických problémů
    \item[+] Užitečné pro učení nových CSS technik
    \item[+] Vynikající pro psaní sémantického HTML
    \item[+] Silná podpora pro accessibility best practices
    \item[+] Generuje moderní, čistý kód bez frameworků
    \item[+] Dobré pochopení kontextu projektu
    \item[-] Občas vyžaduje upřesnění požadavků
\end{itemize}

\subsubsection{Code review -- GitHub Copilot v PR (7/10)}
\begin{itemize}
    \item[+] Automatická kontrola kódu před merge
    \item[+] Identifikace potenciálních problémů
    \item[+] Navrhuje vylepšení a best practices
    \item[+] Přidává vrstvu validace bez manuální práce
    \item[-] Někdy generuje false positive připomínky
    \item[-] Vyžaduje manuální filtrování doporučení
\end{itemize}

\subsection{Přesnost a spolehlivost}

\begin{table}[h]
\centering
\begin{tabular}{|l|c|p{8cm}|}
\hline
\textbf{Oblast} & \textbf{Hodnocení} & \textbf{Poznámky} \\
\hline
Content (GPT 5.1) & 9/10 & Vysoká kvalita textů, minimální úpravy \\
\hline
Images (Web search) & 8/10 & Profesionální fotografie, ale časově náročné hledání \\
\hline
Code (Claude Code) & 8/10 & Moderní, přístupný kód s dobrým pochopením kontextu \\
\hline
Review (Copilot) & 7/10 & Užitečné připomínky, ale vyžaduje filtrování \\
\hline
\end{tabular}
\caption{Hodnocení použitých nástrojů}
\end{table}

\subsection{Doporučení pro použití AI}

\subsubsection{Co dělat}
\begin{itemize}
    \item Používat AI pro brainstorming a inspiraci
    \item Generovat placeholder obsah
    \item Rychlé řešení jednoduchých problémů
    \item Učení se nových technik a konceptů
    \item \textbf{VŽDY} kontrolovat a upravovat AI výstupy
\end{itemize}

\subsubsection{Co nedělat}
\begin{itemize}
    \item Slepě kopírovat AI kód bez pochopení
    \item Spoléhat se na AI pro kritické části projektu
    \item Používat AI generovaný kód bez testování accessibility
    \item Ignorovat best practices kvůli AI návrhům
\end{itemize}

\subsection{Závěr}
AI nástroje byly cenným pomocníkem, ale lidský dohled a odbornost
zůstaly klíčové pro kvalitu projektu. Projekt ironicky demonstruje
správný přístup k AI -- využít ji jako nástroj, ne jako náhradu
fundamentálních dovedností.
