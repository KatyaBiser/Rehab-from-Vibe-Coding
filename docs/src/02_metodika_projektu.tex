\chapter{Metodika projektu}

\section{Postup při tvorbě projektu}

\subsection{Fáze 1: Plánování a návrh (týden 1-2)}
\begin{itemize}
    \item Brainstorming nápadů pro téma rehabilitace
    \item Výběr konceptu ``Rehab from Vibe Coding''
    \item Definice struktury webu a potřebných stránek
    \item Rozdělení rolí v týmu (4 osoby):
    \begin{itemize}
        \item \textbf{Persona A:} Vedoucí projektu, HTML struktura hlavní stránky
        \item \textbf{Persona B:} CSS architekt, responzivní design, Grid layout
        \item \textbf{Persona C:} Content creator, detailní stránky terapeutů
        \item \textbf{Persona D:} Accessibility specialist, testování přístupnosti
    \end{itemize}
\end{itemize}

\subsection{Fáze 2: Tvorba prototypu (týden 2-3)}
\begin{itemize}
    \item Wireframing základní struktury v Figma/tužka a papír
    \item Diskuse nad layoutem a informační architekturou
    \item Vytvoření barevné palety (medical/clean theme)
    \item Návrh typografie a spacing systému
    \item Schválení struktury a designu v týmu
\end{itemize}

\subsection{Fáze 3: Generování obsahu (týden 3-4)}
\begin{itemize}
    \item Využití AI nástrojů pro generování textového obsahu
    \item Generování obrázků pomocí AI (DALL-E, Midjourney)
    \item Vytvoření profilů 8 fiktivních terapeutů
    \item Napsání biografií, specializací a recenzí
    \item Příprava tabulky příznaků a glosáře pojmů
\end{itemize}

\subsection{Fáze 4: HTML implementace (týden 4-6)}
\begin{itemize}
    \item Vytvoření základní HTML struktury index.html
    \item Implementace sémantických elementů (article, section, aside)
    \item Tvorba stránky se seznamem terapeutů
    \item Implementace 8 detailních profilů terapeutů
    \item Zajištění správné hierarchie nadpisů
    \item Přidání meta tagů a favicon
\end{itemize}

\subsection{Fáze 5: CSS styling (týden 6-8)}
\begin{itemize}
    \item Vytvoření modulární CSS architektury
    \item Implementace CSS Custom Properties (variables.css)
    \item Base styles (reset, typography)
    \item Layout komponenty (grid, flexbox, header, footer)
    \item Styling jednotlivých komponent (cards, buttons, tables)
    \item Responzivní media queries (mobile-first)
\end{itemize}

\subsection{Fáze 6: Testování a optimalizace (týden 8-9)}
\begin{itemize}
    \item Testování responzivity na různých zařízeních
    \item Kontrola accessibility pomocí WAVE a axe DevTools
    \item Validace HTML (W3C Validator)
    \item Validace CSS (W3C CSS Validator)
    \item Testování ve více prohlížečích (Chrome, Firefox, Safari)
    \item Performance optimalizace (image compression)
\end{itemize}

\subsection{Fáze 7: Deployment a dokumentace (týden 9-10)}
\begin{itemize}
    \item Nahrání projektu na GitHub
    \item Aktivace GitHub Pages
    \item Napsání README.md
    \item Příprava finální dokumentace pro odevzdání
    \item Kontrola všech požadavků projektu
\end{itemize}

\section{Software, nástroje a IDE}

\subsection{Vývojové prostředí}
\begin{description}
    \item[Visual Studio Code] Hlavní IDE
    \begin{itemize}
        \item Rozšíření: Live Server pro lokální preview
        \item Rozšíření: HTML CSS Support pro autocomplete
        \item Rozšíření: Prettier pro formátování kódu
        \item Rozšíření: axe Accessibility Linter
    \end{itemize}
\end{description}

\subsection{Design a prototypování}
\begin{itemize}
    \item Figma -- wireframing a mockupy (používala Persona B)
    \item Papír a tužka -- rychlé skici layoutu
\end{itemize}

\subsection{Správa verzí}
\begin{itemize}
    \item Git -- verzování kódu
    \item GitHub -- hosting repository a collaboration
    \item GitHub Desktop -- GUI pro Git (pro méně technické členy týmu)
\end{itemize}

\subsection{Generování obsahu}
\begin{itemize}
    \item ChatGPT -- generování textů, biografií, recenzí
    \item Claude AI -- asistence při psaní obsahu
    \item DALL-E 3 / Midjourney -- generování obrázků terapeutů
\end{itemize}

\subsection{Testování a validace}
\begin{itemize}
    \item W3C HTML Validator -- kontrola validity HTML
    \item W3C CSS Validator -- kontrola validity CSS
    \item WAVE Web Accessibility Evaluation Tool
    \item axe DevTools -- accessibility testing v prohlížeči
    \item Lighthouse -- performance a accessibility audit
    \item BrowserStack -- testování v různých prohlížečích
\end{itemize}

\subsection{Grafické úpravy}
\begin{itemize}
    \item Adobe Photoshop -- úprava AI generovaných obrázků
    \item TinyPNG -- komprese obrázků pro web
\end{itemize}

\subsection{Prohlížeče pro testování}
\begin{itemize}
    \item Google Chrome + DevTools
    \item Mozilla Firefox + Developer Tools
    \item Safari (macOS)
    \item Microsoft Edge
\end{itemize}

\section{Využití AI nástrojů při tvorbě projektu}

\subsection{Generování textového obsahu}

\subsubsection{Použité nástroje}
\begin{itemize}
    \item ChatGPT-4
    \item Claude 3.5 Sonnet
\end{itemize}

\subsubsection{Jak byly využity}
\begin{itemize}
    \item Generování humorných biografií terapeutů
    \item Vytváření konzistentního tone of voice
    \item Psaní recenzí od fiktivních klientů
    \item Generování tabulky příznaků AI závislosti
    \item Tvorba glosáře odborných pojmů
\end{itemize}

\subsubsection{Příklady promptů}
\begin{lstlisting}[style=htmlstyle, caption={Prompt pro tvorbu profilu terapeuta}]
Vytvor humorny profil fiktivniho terapeuta Dr. Emma HTML,
ktera se specializuje na lecbu vyvojaru zavislych na pouzivani
<div> elementu misto semantickeho HTML. Zahrn biografii,
specializace a recenze od klientu.
Ton: satiricky, ale empaticky.
\end{lstlisting}

\subsection{Generování vizuálního obsahu}

\subsubsection{Použité nástroje}
\begin{itemize}
    \item DALL-E 3 (via ChatGPT Plus)
    \item Midjourney v6
\end{itemize}

\subsubsection{Jak byly využity}
\begin{itemize}
    \item Generování portrétů 8 fiktivních terapeutů
    \item Vytváření ilustrací pro sekce hlavní stránky
    \item Design hero images
\end{itemize}

\subsubsection{Příklady promptů}
\begin{lstlisting}[style=htmlstyle, caption={Prompt pro generování portrétu}]
Professional portrait photo of a friendly female therapist
in her 40s, wearing business casual attire, warm smile,
modern medical office background, high quality,
corporate photography style
\end{lstlisting}

\subsection{Asistence při psaní kódu}

\subsubsection{Použité nástroje}
\begin{itemize}
    \item GitHub Copilot (vypnuto pro většinu práce)
    \item ChatGPT pro specifické problémy
\end{itemize}

\subsubsection{Jak byly využity}
\begin{itemize}
    \item Řešení CSS Grid problémů s layout
    \item Kontrola accessibility best practices
    \item Generování CSS custom properties struktur
    \item Návody na responzivní design patterns
\end{itemize}

\textbf{POZOR:} Většina HTML/CSS byla napsána ručně bez AI asistence,
aby projekt skutečně demonstroval fundamentální dovednosti.

\section{Kritické zhodnocení AI nástrojů}

\subsection{Přínosy}

\subsubsection{Content generation (9/10)}
\begin{itemize}
    \item[+] Rychlé vytváření konzistentního obsahu
    \item[+] Kreativní nápady pro humorné texty
    \item[+] Úspora času při psaní opakujícího se obsahu
    \item[+] Schopnost generovat různé varianty textu
\end{itemize}

\subsubsection{Image generation (7/10)}
\begin{itemize}
    \item[+] Rychlé prototypování vizuálů
    \item[+] Konzistentní stylový směr
    \item[+] Levnější než stock fotografie
    \item[+] Možnost iterace a úprav
\end{itemize}

\subsubsection{Code assistance (6/10)}
\begin{itemize}
    \item[+] Rychlé řešení specifických problémů
    \item[+] Užitečné pro učení nových CSS technik
    \item[+] Dobré pro boilerplate kód
    \item[-] Často navrhuje frameworky místo custom CSS
    \item[-] Někdy generuje neoptimální nebo nepřístupný kód
\end{itemize}

\subsection{Přesnost a spolehlivost}

\begin{table}[h]
\centering
\begin{tabular}{|l|c|p{8cm}|}
\hline
\textbf{Oblast} & \textbf{Hodnocení} & \textbf{Poznámky} \\
\hline
Content & 8/10 & Gramaticky správné, ale někdy vyžaduje úpravu \\
\hline
Images & 6/10 & Vizuálně konzistentní, ale problémy s detaily \\
\hline
Code & 5/10 & Syntakticky správný, ale často zastaralý \\
\hline
\end{tabular}
\caption{Hodnocení přesnosti AI nástrojů}
\end{table}

\subsection{Doporučení pro použití AI}

\subsubsection{Co dělat}
\begin{itemize}
    \item Používat AI pro brainstorming a inspiraci
    \item Generovat placeholder obsah
    \item Rychlé řešení jednoduchých problémů
    \item Učení se nových technik a konceptů
    \item \textbf{VŽDY} kontrolovat a upravovat AI výstupy
\end{itemize}

\subsubsection{Co nedělat}
\begin{itemize}
    \item Slepě kopírovat AI kód bez pochopení
    \item Spoléhat se na AI pro kritické části projektu
    \item Používat AI generovaný kód bez testování accessibility
    \item Ignorovat best practices kvůli AI návrhům
\end{itemize}

\subsection{Závěr}
AI nástroje byly cenným pomocníkem, ale lidský dohled a odbornost
zůstaly klíčové pro kvalitu projektu. Projekt ironicky demonstruje
správný přístup k AI -- využít ji jako nástroj, ne jako náhradu
fundamentálních dovedností.

\section{Rozdělení práce v týmu}

\subsection{Persona A (Vedoucí projektu, HTML struktura)}
\begin{itemize}
    \item Koordinace týmu a rozdělení úkolů
    \item Hlavní HTML struktura index.html
    \item Implementace sémantických elementů
    \item Code review HTML souborů
    \item Příprava Git repository a GitHub Pages deployment
\end{itemize}

\subsection{Persona B (CSS architekt, responzivní design)}
\begin{itemize}
    \item Návrh CSS architektury a modulární struktury
    \item Implementace CSS Custom Properties
    \item CSS Grid a Flexbox layouts
    \item Responzivní media queries (mobile-first)
    \item Performance optimalizace CSS
\end{itemize}

\subsection{Persona C (Content creator, detailní stránky)}
\begin{itemize}
    \item Generování obsahu pomocí AI nástrojů
    \item Tvorba 8 detailních profilů terapeutů
    \item Implementace HTML pro detailní stránky
    \item Psaní textů pro všechny sekce webu
    \item Příprava tabulek a glosáře
\end{itemize}

\subsection{Persona D (Accessibility specialist)}
\begin{itemize}
    \item Implementace ARIA labels a roles
    \item Accessibility testing (WAVE, axe)
    \item Kontrola barevných kontrastů
    \item Testování s asistivními technologiemi
    \item Focus states a keyboard navigation
    \item Finální accessibility audit
\end{itemize}

\subsection{Všichni členové týmu}
\begin{itemize}
    \item Pravidelné týmové schůzky (weekly standups)
    \item Code review navzájem
    \item Testování responzivity na vlastních zařízeních
    \item Společná finální kontrola před odevzdáním
    \item Příprava dokumentace
\end{itemize}
