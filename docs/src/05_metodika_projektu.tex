\chapter{Metodika projektu}

\section{Postup při tvorbě projektu}
\begin{itemize}
  \item Agile light režim: týdenní sprinty, plánování v pondělí, mid-week sync a krátká retro na konci týdne.
  \item Fáze: (1) rešerše požadavků a hi-fi Figma prototyp; (2) převod na HTML/CSS; (3) ladění responzivity a WCAG 2.0 AA; (4) interní review a finální validace.
  \item Úkoly psané jako krátké user stories s akceptačními kritérii, uzavírané po vizuální a kódové kontrole.
  \item Pravidelné validace: W3C Nu Html Checker, vizuální kontrola proti prototypu, manuální test keyboard navigace a kontrastu.
\end{itemize}

\section{Software, nástroje, IDE}
\begin{itemize}
  \item VS Code jako hlavní IDE (rozšíření: Live Server pro náhled, Prettier pro formátování).
  \item Git a GitHub pro verzování a PR review; GitHub Actions pro automatický deploy na GitHub Pages.
  \item Figma pro hi-fi prototyp a handoff designu.
  \item Kontrola kvality: W3C Nu Html Checker, Chrome DevTools (Lighthouse/Accessibility panel) pro ověření kontrastu a focus stavů.
\end{itemize}
