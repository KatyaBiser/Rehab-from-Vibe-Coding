\chapter{Využití generativní AI}

\section{Úvod}

\begin{itemize}
  \item Generativní AI jsme použili pro texty a pro asistenci při kódu.
  \item Většina HTML/CSS je psaná ručně; AI urychlila dílčí úkoly, ale nenahradila fundamentální dovednosti.
\end{itemize}

\section{Nástroje pro generování textového obsahu}

\subsection{GPT 5.1}

\begin{itemize}
  \item Generování textů pro web: konzistentní tone of voice, satira + empatie.
  \item Vytvoření tabulkových dat a marketingových textů s minimem editací.
  \item Pokrytí 9 profilů terapeutů během ~15 minut.
\end{itemize}

\subsubsection{Příklad promptu}

\begin{lstlisting}[style=promptstyle]
Vytvor humorny profil fiktivniho terapeuta pro web "Rehab
from Vibe Coding" - rehabilitacni centrum pro vyvojare
zavisla na AI nastroje.

Jmeno a specializace: Vymysli kreativni jmeno a prijmeni
souvisejici s webovym vyvojem (napr. semanticke HTML,
CSS vlastnosti, web accessibility, web technologie).

Struktura profilu:
1. Biography (3 odstavce):
   - Pozadi a motivace terapeuta
   - Osobni pristup k lecbe
   - Predchozi zkusenosti

2. Areas of Expertise (5-7 bodu):
   - Specificke oblasti terapie
   - Humorny pristup k webovym technologiim

3. Selected Publications (4-5 fiktivnich publikaci):
   - Nazvy clanku/knih s vtipnymi odkazy na web dev
   - Fiktivni casopisy/vydavatele

4. Education & Certifications (4-5 bodu):
   - Fiktivni univerzity a certifikace
   - Creativni nazvy spojena s technologiemi

5. Treatment Approach (uvodni text + 5-6 bodu):
   - Terapeuticke metody s vtipnou terminologii

6. What Clients Say (2 testimonials):
   - Fiktivni recenze od vyzdravelych vyvojaru
   - Format: citace + jmeno klienta

Ton: satiricky a humorny, ale s empatii k vyvojarum.
Musi byt zabavny, ale ne urazazejici.
Delka: poucujuci, ale ne prilis dlouhy.
\end{lstlisting}

\begin{itemize}
  \item Zrychlení tvorby obsahu (9 profilů za ~15 minut místo hodin).
  \item Silný brainstorming pro humor + konzistentní styl.
  \item Výstupy byly použitelné bez opakovaných promptů.
\end{itemize}

\section{Nástroje pro asistenci při psaní kódu}

\subsection{Claude Code (Sonnet 4.5)}

\begin{itemize}
  \item Selektivní použití pro CSS/Grid/Flex konzultace a WCAG best practices.
  \item Pomoc se strukturou custom properties a Mobile-first breakpoints.
  \item Refaktoringové tipy: odstranění duplicit, lepší pojmenování.
\end{itemize}

\subsubsection{Příklad promptu}

\begin{lstlisting}[style=promptstyle]
Mam CSS Grid s 9 kartami (4 sloupce na desktopu). Chci
vytvorit zigzag pattern, kde nektery karty zabiraji
2 sloupce misto 1:

- Radek 1: Karta #1 span 2 (vlevo), karty #2,3 normalne
- Radek 2: Karty #4,5 normalne, karta #6 span 2 (vpravo)
- Radek 3: Karta #7 span 2 (vlevo), karty #8,9 normalne

Pls objasni jak to udelat - jake CSS selektory pouzit,
jake vlastnosti nastavit.
\end{lstlisting}

\begin{itemize}
  \item Rychlé řešení CSS problémů (minuty místo hodin) s vysvětlením „proč“.
  \item Silná opora v accessibility/WCAG.
  \item Limit: někdy nutné přesněji zadat, chybí vizuální kontext.
\end{itemize}

\subsection{GitHub Copilot (Code Review)}

\begin{itemize}
  \item Automatické PR review: hledání chyb (alt texty, sémantika, naming) a návrhy na vylepšení.
  \item Šetří čas před merge; jediný drawback je čekání ~5 minut na výsledek.
\end{itemize}

\begin{figure}[H]
\centering
\includegraphics[width=0.8\textwidth]{screens/03/01-copilot-check.jpg}
\caption{GitHub Copilot code review -- identifikace problému se sémantickou strukturou HTML a návrh vylepšení s použitím \texttt{<dl>} místo odděleného \texttt{<li>} elementu}
\end{figure}

\section{Kritické zhodnocení AI nástrojů}

\subsection{Přesnost a spolehlivost}

Následující tabulka shrnuje hodnocení jednotlivých AI nástrojů:

\begin{table}[ht]
\centering
\begin{tabular}{|l|c|p{7cm}|}
\hline
\textbf{Nástroj} & \textbf{Hodnocení} & \textbf{Poznámky} \\
\hline
GPT 5.1 & 10/10 & Vysoká kvalita textů; všechny výstupy byly hned použitelné bez nutnosti dalších úprav; výborná konzistence tónu \\
\hline
Claude Code & 9/10 & Moderní, přístupný kód s dobrým pochopením kontextu; občas vyžaduje iteraci \\
\hline
GitHub Copilot & 10/10 & Rychlé PR review; jen čekání ~5 minut na dokončení, pořád rychlejší než člověk \\
\hline
\end{tabular}
\caption{Souhrnné hodnocení generativních AI nástrojů}
\end{table}

\subsection{Přínosy a omezení}

\begin{itemize}
  \item Kombinace: GPT 5.1 pro obsah, Claude Code pro techniku, Copilot pro PR review \textrightarrow{} kratší editační cykly a konzistence.
  \item AI pokryla rutinní úkoly, tým se soustředil na design, přístupnost a ladění.
  \item Nevýhody: Copilot má latenci ~5 min; modely bez vizuálního kontextu \textrightarrow{} nutná finální lidská kontrola.
\end{itemize}

\subsection{Doporučení pro použití AI}

\begin{itemize}
  \item Doporučení: AI pro brainstorming, placeholdery, konzultace specifických problémů, učení se s vysvětlením, vždy validovat/testovat výstupy.
  \item Co nedělat: slepé kopírování bez porozumění, spoléhat se na AI u kritických částí (security, accessibility), ignorovat best practices kvůli rychlosti.
\end{itemize}

\section{Závěr}

\begin{itemize}
  \item AI pomohla, ale klíčový zůstal lidský dohled, kritické myšlení a ručně psaný kód.
  \item GPT 5.1 zrychlil texty, Claude Code řešil CSS/aksesibility otázky, Copilot kontroloval PR.
  \item Poučení: AI je nejlepší jako doplněk, ne náhrada; to je i hlavní message projektu.
\end{itemize}
