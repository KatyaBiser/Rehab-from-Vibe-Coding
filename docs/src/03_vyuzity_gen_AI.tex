\chapter{Využití generativní AI}

\section{Úvod}

Při tvorbě projektu byly využity nástroje generativní AI pro dvě hlavní oblasti: generování textového obsahu a asistenci při psaní kódu. Přestože AI významně urychlily některé aspekty vývoje, většina HTML a CSS byla napsána ručně bez AI asistence, což odráží hlavní poselství projektu -- demonstraci fundamentálních webových dovedností.

\section{Nástroje pro generování textového obsahu}

\subsection{GPT 5.1}

GPT 5.1 od OpenAI byl použit primárně pro vytváření textového obsahu webu. Model generoval konzistentní tone of voice napříč celým webem, vytvořil strukturovaný obsah kombinující satirický humor s empatickým tónem a pomohl s generováním tabulkových dat a marketingových textů.

\subsubsection{Příklad promptu}

\begin{lstlisting}[style=promptstyle]
Vytvor humorny profil fiktivniho terapeuta pro web "Rehab
from Vibe Coding" - rehabilitacni centrum pro vyvojare
zavisla na AI nastroje.

Jmeno a specializace: Vymysli kreativni jmeno a prijmeni
souvisejici s webovym vyvojem (napr. semanticke HTML,
CSS vlastnosti, JavaScript koncepty, web technologie).

Struktura profilu:
1. Biography (3 odstavce):
   - Pozadi a motivace terapeuta
   - Osobni pristup k lecbe
   - Predchozi zkusenosti

2. Areas of Expertise (5-7 bodu):
   - Specificke oblasti terapie
   - Humorny pristup k webovym technologiim

3. Selected Publications (4-5 fiktivnich publikaci):
   - Nazvy clanku/knih s vtipnymi odkazy na web dev
   - Fiktivni casopisy/vydavatele

4. Education & Certifications (4-5 bodu):
   - Fiktivni univerzity a certifikace
   - Creativni nazvy spojena s technologiemi

5. Treatment Approach (uvodni text + 5-6 bodu):
   - Terapeuticke metody s vtipnou terminologii

6. What Clients Say (2 testimonials):
   - Fiktivni recenze od vyzdravelych vyvojaru
   - Format: citace + jmeno klienta

Ton: satiricky a humorny, ale s empatii k vyvojarum.
Musi byt zabavny, ale ne urazazejici.
Delka: poucujuci, ale ne prilis dlouhy.
\end{lstlisting}

GPT 5.1 umožnil rychlé vytváření konzistentního obsahu -- všech devět profilů terapeutů bylo vygenerováno během 15 minut místo odhadovaných 4-6 hodin ručního psaní. Nástroj generuje kreativní nápady pro humorné texty a udržuje konzistentní styl napříč všemi texty. Hlavními omezeními jsou občasné generické formulace vyžadující manuální úpravu, tendence opakovat podobné fráze napříč profily a potřeba několika iterací promptu pro dosažení požadovaného výsledku.

\section{Nástroje pro asistenci při psaní kódu}

\subsection{Claude Code (Sonnet 4.5)}

Claude Code od Anthropic byl použit selektivně pro řešení specifických problémů a kontrolu best practices. Nástroj pomáhal s konfigurací responzivního CSS Grid layoutu pro karty terapeutů s transformací z jednoho na tři sloupce, kontrolou accessibility best practices (ověření ARIA atributů, skip links, screen reader textů), generováním struktury CSS custom properties v souboru \texttt{assets/css/style.css} s konzistentním pojmenováním, konzultací správných přístupů k Mobile-first designu a breakpointům a refaktoringem kódu -- identifikací duplicit a navrhováním vylepšení.

\subsubsection{Příklad promptu}

\begin{lstlisting}[style=promptstyle]
Mam CSS Grid s 9 kartami (4 sloupce na desktopu). Chci
vytvorit zigzag pattern, kde nektery karty zabiraji
2 sloupce misto 1:

- Radek 1: Karta #1 span 2 (vlevo), karty #2,3 normalne
- Radek 2: Karty #4,5 normalne, karta #6 span 2 (vpravo)
- Radek 3: Karta #7 span 2 (vlevo), karty #8,9 normalne

Pls objasni jak to udelat - jake CSS selektory pouzit,
jake vlastnosti nastavit.
\end{lstlisting}

Claude Code poskytuje velmi rychlé řešení technických problémů (CSS Grid konfigurace za minuty místo hodin testování), je užitečný pro učení nových CSS technik s vysvětlením ``proč'' funguje daný kód a má silnou podporu pro accessibility best practices s referencemi na WCAG guidelines. Omezením je občasná potřeba upřesnění požadavků pro ideální výsledek, navrhované řešení není vždy nejoptimálnější a model nemá přístup k prohlížeči pro vizuální ověření.

\subsection{GitHub Copilot (Code Review)}

GitHub Copilot v pull requests poskytuje automatickou code review s identifikací potenciálních problémů, bezpečnostních rizik a návrhů na vylepšení. Nástroj byl použit pro automatickou kontrolu kódu před merge, identifikaci chybějících alt textů a nekonzistentního naming a navrhování vylepšení podle best practices. Výhodou je úspora času při review procesu a identifikace drobných chyb, nevýhodou jsou občasné false positive připomínky a nutnost manuálního filtrování doporučení.

\begin{figure}[H]
\centering
\includegraphics[width=\textwidth]{screens/03/01-copilot-check.jpg}
\caption{GitHub Copilot code review -- identifikace problému se sémantickou strukturou HTML a návrh vylepšení s použitím \texttt{<dl>} místo odděleného \texttt{<li>} elementu}
\end{figure}

\section{Kritické zhodnocení AI nástrojů}

\subsection{Přesnost a spolehlivost}

Následující tabulka shrnuje hodnocení jednotlivých AI nástrojů:

\begin{table}[ht]
\centering
\begin{tabular}{|l|c|p{7cm}|}
\hline
\textbf{Nástroj} & \textbf{Hodnocení} & \textbf{Poznámky} \\
\hline
GPT 5.1 & 9/10 & Vysoká kvalita textů s minimálními úpravami; výborná konzistence tónu \\
\hline
Claude Code & 8/10 & Moderní, přístupný kód s dobrým pochopením kontextu; občas vyžaduje iteraci \\
\hline
GitHub Copilot & 7/10 & Užitečné připomínky, ale vyžaduje manuální filtrování návrhů \\
\hline
\end{tabular}
\caption{Souhrnné hodnocení generativních AI nástrojů}
\end{table}

\subsection{Přínosy a omezení}

AI nástroje významně zkrátily čas potřebný pro vytvoření obsahu. Generování devíti profilů terapeutů zabralo 30 minut místo odhadovaných 4-6 hodin. GPT 5.1 zajistil jednotný tone of voice napříč všemi texty a poskytl kreativní nápady pro humorné názvy a koncepty. Asistence od Claude Code měla pedagogickou hodnotu -- vysvětlovala ``proč'' funguje dané řešení, což pomohlo pochopit CSS Grid, Flexbox a accessibility best practices.

Hlavní omezení zahrnují občasnou potřebu manuální úpravy generovaných textů pro přidání specifičnosti, nedostatek kontextu -- AI modely nemají přístup k vizuálnímu zobrazení webu ani širšímu kontextu projektu, riziko závislosti -- přílišné spoléhání se na AI může vést ke ztrátě fundamentálních dovedností (přesně problém, který projekt satiricky kritizuje), a nutnost validace -- GPT 5.1 občas generuje fakticky nesprávné informace (hallucinations) a Claude Code navrhuje řešení, která nebyla vždy optimální.

\subsection{Doporučení pro použití AI}

Na základě zkušeností lze doporučit následující best practices: používat AI pro brainstorming a inspiraci (nikoliv jako konečné řešení), generovat placeholder obsah pro rychlé prototypování, konzultovat specifické technické problémy místo kopírování celých komponent, učit se nových technik s vysvětlením od AI a VŽDY kontrolovat a upravovat AI výstupy. Co nedělat: slepě kopírovat AI kód bez pochopení, spoléhat se na AI pro kritické části projektu (security, accessibility), používat AI kód bez testování, ignorovat best practices kvůli ``rychlejším'' návrhům a nahrazovat fundamentální dovednosti AI asistencí.

\section{Závěr}

AI nástroje byly cenným pomocníkem při tvorbě projektu, ale lidský dohled, kritické myšlení a odbornost zůstaly klíčové pro kvalitu výsledku. GPT 5.1 urychlil tvorbu textového obsahu, Claude Code pomohl s technickými problémy a učením best practices, ale většina kódu byla napsána ručně pro demonstraci základních HTML/CSS dovedností. Klíčovým poznatkem je, že AI asistence je nejefektivnější když doplňuje, ne nahrazuje, lidské dovednosti -- přesně tím, co projekt ``Rehab from Vibe Coding'' propaguje.
