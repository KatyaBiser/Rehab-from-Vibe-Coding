\chapter{Požadavky na implementaci}
\label{chap:pozadavky}

\section{Popis webových stránek}

Web se skládá z jedenácti HTML stránek: úvodní stránka, seznam terapeutů a devět detailních profilů. Každá stránka je navržena s důrazem na konzistenci designu, přístupnost a responzivní chování.

\subsection{Úvodní stránka (index.html)}

Úvodní stránka obsahuje hero sekci s hlavním nadpisem \textit{``Tired of AI Writing Code for You?''}, sekci o misi centra, interaktivní tabulku příznaků AI závislosti s osmi řádky symptomů a doporučenými léčebnými programy, glosář odborných pojmů (používá sémantické elementy \texttt{<dl>}, \texttt{<dt>}, \texttt{<dd>}), přehled terapeutických programů, testimonials od fiktivních klientů v \texttt{<blockquote>} elementech a footer s kontakty organizovaný pomocí CSS Grid layoutu do čtyř sloupců.

\begin{figure}[H]
\centering
\includegraphics[width=\textwidth]{screens/02/01-hero-section.png}
\caption{Hero sekce úvodní stránky}
\end{figure}

\begin{figure}[H]
\centering
\includegraphics[width=\textwidth]{screens/02/02-symptoms-table.png}
\caption{Tabulka příznaků AI závislosti}
\end{figure}

\begin{figure}[H]
\centering
\includegraphics[width=\textwidth]{screens/02/03-testimonials.png}
\caption{Testimonials sekce}
\end{figure}

\subsection{Seznam terapeutů (pages/therapists/list.html)}

Stránka prezentuje devět terapeutů v responzivním grid layoutu s úvodním textem a devíti kartami. Každá karta obsahuje portrét, jméno, specializaci, stručný popis a odkaz na detailní profil. Tři terapeuti mají badge \textit{``Lead Therapist''}. Grid se přizpůsobuje šířce obrazovky: jeden sloupec na mobilech, dva na tabletech, tři na desktopech.

\begin{figure}[H]
\centering
\includegraphics[width=\textwidth]{screens/02/04-therapists-grid-desktop.png}
\caption{Grid layout se všemi 9 terapeuty (desktop view)}
\end{figure}

\subsection{Detailní profily terapeutů}

Každý terapeut má dedikovanou HTML stránku s breadcrumb navigací, header s portrétem a jménem, biografií, oblastmi expertízy, vybranými publikacemi, vzděláním a certifikacemi, terapeutickým přístupem, recenzemi klientů v \texttt{<blockquote>} elementech a call-to-action sekcí pro rezervaci.

\begin{figure}[H]
\centering
\includegraphics[width=\textwidth]{screens/02/06-therapist-detail.png}
\caption{Detailní profil terapeuta}
\end{figure}

\section{Chování a interaktivita stránek}

\subsection{Responzivní layout}

Web implementuje Mobile-first přístup s třemi breakpointy: Mobile (<768px) s výchozími styly a jedním sloupcem, Tablet ($\geq$768px) s dvousloupcovými layouty a horizontální navigací, Desktop ($\geq$1024px) s tří a čtyřsloupcovými layouty a maximální šířkou 1200px. Grid layout terapeutů se transformuje z jednoho sloupce přes dva na tři sloupce. Footer grid mění strukturu z jednoho na čtyři sloupce.

% SCREENSHOT 7: Srovnání mobile vs desktop view (split screen)
% Popis: Stejná stránka zobrazená vedle sebe na mobilu a desktopu
% Soubor: screenshot-07-responsive-comparison.png

\subsection{Navigace}

Na mobilních zařízeních je menu defaultně skryté s hamburger tlačítkem. JavaScript (\texttt{navigation.js}) přidává třídu \texttt{nav\_\_list--active} při kliknutí, což spouští slide-in animaci z pravé strany a nastavuje \texttt{aria-expanded="true"}. Menu se automaticky zavírá při kliknutí na odkaz, mimo menu nebo změně velikosti okna na tablet/desktop breakpoint. Na větších obrazovkách je menu vždy viditelné jako horizontální seznam.

\subsection{Vyhledávání a interaktivní efekty}

Vyhledávací pole je implementováno jako dropdown aktivovaný tlačítkem s ikonou lupy. JavaScript přidává třídu \texttt{search-dropdown--active}, zobrazuje dropdown s CSS transition a nastavuje automatický focus na input pole. Všechny interaktivní elementy mají hover efekty (karty s \texttt{transform: translateY(-5px)}, tlačítka s \texttt{transform: scale(1.02)}) a výrazné focus stavy s modrým obrysem pro klávesnicovou navigaci. Skip link je viditelný při focusu.

\section{HTML požadavky}

\subsection{Struktura dokumentu}

Všechny stránky používají HTML5 standard s \texttt{<!DOCTYPE html>}, \texttt{<html lang="en">}, meta tagy pro kódování UTF-8, viewport pro responzivitu, description, keywords a title specifický pro každou stránku. Favicon a apple-touch-icon jsou definovány. Všechny stránky linkují jeden centrální CSS soubor s relativním adresováním.

\subsection{Sémantické elementy}

Projekt využívá sémantické HTML5 elementy: \texttt{<header>} pro hlavičku s logem a navigací, \texttt{<nav>} s \texttt{aria-label}, \texttt{<main id="main-content">} jako target pro skip link, \texttt{<section>} pro tematické sekce, \texttt{<article>} pro karty a profily, \texttt{<aside>} pro testimonials, \texttt{<footer>} pro patičku. Dodržována je správná hierarchie nadpisů s jedním \texttt{<h1>} na stránku. Tabulka používá \texttt{<thead>}, \texttt{<tbody>}, \texttt{<th scope="col">}. Glosář využívá \texttt{<dl>}, \texttt{<dt>}, \texttt{<dd>}.

\subsection{Přístupnost (WCAG 2.0 Level AA)}

Web splňuje WCAG 2.0 Level AA požadavky. Skip link \texttt{<a href="\#main-content"} \texttt{class="skip-link">} je umístěn na začátku každé stránky. Použité jsou ARIA atributy (\texttt{aria-label}, \texttt{aria-expanded}, \texttt{aria-current="page"}). Screen reader texty používají třídu \texttt{sr-only}. Všechny obrázky mají smysluplné alt texty. Barevný kontrast je vyšší než 7:1 (tmavě šedá \texttt{\#212529} na bílém pozadí). Výrazné focus stavy zajišťují klávesnicovou navigaci.

\section{CSS požadavky}

\subsection{Architektura CSS}

Projekt používá 100\% vlastní CSS bez frameworků. Veškerý CSS je organizován v jediném souboru \texttt{assets/css/style.css}, který obsahuje všechny styly strukturované do logických sekcí s komentáři: CSS Variables, Reset \& Base Styles, Typography, Layout (Container, Header, Footer, Grid), Components (Buttons, Cards, Search, Hero, Tables, Breadcrumbs, Testimonials), Pages (Detail, Sections), Utilities (Accessibility, Helpers, Glossary) a Responsive Design. Celková velikost je 1249 řádků optimalizovaného kódu.

\subsection{CSS Custom Properties}

{\sloppy
Systém CSS Custom Properties je definován v \texttt{:root} sekci souboru \texttt{assets/css/}\allowbreak\texttt{style.css}.
\par}

Pokrývá barevnou paletu (\texttt{--primary-color}, \texttt{--secondary-color}, \texttt{--accent-color}, atd.) a typografii (\texttt{--font-primary}, \texttt{--font-heading}, velikosti písma).

Spacing (\texttt{--spacing-xs} až \texttt{--spacing-xxl} od 0.5rem do 4rem) je oddělená skupina proměnných. Layout obsahuje \texttt{--max-width-container} a \texttt{--border-radius}. Další proměnné: \texttt{--box-shadow}, \texttt{--transition-speed}. Proměnné zajišťují konzistenci a snadné globální úpravy designu.

\subsection{CSS Grid layout}

CSS Grid je použit pro layout seznamu terapeutů (\texttt{.grid--therapists}) a footer (\texttt{.footer\_\_grid}). Grid terapeutů obsahuje devět karet s responzivními sloupci: \texttt{grid-template-columns: 1fr} na mobilech, \texttt{repeat(2, 1fr)} na tabletech, \texttt{repeat(3, 1fr)} na desktopech. Gap mezi položkami je \texttt{var(--spacing-md)}. Footer grid používá čtyři sloupce na desktopech s adaptací na menší počet sloupců. Grid nabízí 2D kontrolu (řádky i sloupce) a jednodušší responzivní úpravy než Flexbox.

\subsection{Flexbox}

Flexbox je použit pro jednodimenzionální layouty: navigace (\texttt{.nav\_\_list} s flexbox pro horizontální/vertikální rozložení), header (\texttt{.header\_\_container} s \texttt{flex-direction: column/row} podle breakpointu), karty (\texttt{.card} s \texttt{flex-direction: column}) a tlačítka (\texttt{.btn} s \texttt{inline-flex; align-items: center; justify-content: center}).

\subsection{Responzivní design}

Mobile-first přístup používá základní styly pro nejmenší obrazovky. Media queries \texttt{@media (min-width: ...)} rozšiřují layout. Techniky zahrnují fluid typography pomocí relative units (rem, em), fluid layouts pomocí procent a CSS Grid/Flexbox. Responsive images s \texttt{max-width: 100\%; height: auto}, conditional visibility pomocí \texttt{display: none} a transformaci navigace z vertikální na horizontální.

\section{JavaScript požadavky}

\subsection{Vanilla JavaScript}

Projekt používá čistý JavaScript bez knihoven. Celý kód je v souboru \texttt{navigation.js} (85 řádků) a zaměřuje se na interaktivitu UI. Kód začíná \texttt{DOMContentLoaded} event listenerem.

Používá moderní metody: \texttt{querySelector}, \texttt{addEventListener}, \texttt{classList.toggle}.

\subsection{Funkcionality}

Mobilní navigace s toggle hamburger menu, animacemi, správou ARIA atributů (\texttt{aria-expanded}), prevencí scrollování pozadí (\texttt{body.style.overflow}) a automatickým zavíráním. Vyhledávání s toggle search dropdown a focus managementem. Responsive listener s \texttt{window.matchMedia} pro detekci změny breakpointu. Správa accessibility s focus stavy, ARIA atributy a keyboard navigation support. Event listener pro dummy odkazy \texttt{href="\#"} preventující default chování.

\section{Vizuální obsah a média}

\subsection{Zdroje fotografií}

Pro získání fotografií terapeutů a ilustračních obrázků byl použit Google Images. Fotografie byly vyhledány pomocí jednoduchých dotazů jako ``lekari fotografie'' pro portréty terapeutů a ``stastni lide v rehabilitaci'' pro ilustrační obrázky na úvodní stránce. Pro každého z devíti terapeutů byla vybrána profesionální fotografie odpovídající specializaci a osobnosti popsané v biografii.

\subsection{Volba mezi AI generovanými a stock fotografiemi}

Přístup s vyhledáním reálných fotografií byl vybrán místo AI generování obrázků (DALL-E, Midjourney) pro zajištění realistického vzhledu skutečných lidí. AI generované portréty často trpí artefakty a ``uncanny valley'' efektem, který by narušil autenticitu prezentace rehabilitačního centra.

\subsection{Licenční poznámka}

Protože se jedná o studentský učební projekt, nebyla řešena otázka licencí použitých fotografií. Fotografie byly použity tak, jak byly nalezeny, bez další úpravy formátu nebo velikosti. V reálném komerčním projektu by bylo nutné řešit autorská práva a licence fotografií, případně použít placené stock fotografie (Unsplash, Pexels) nebo AI generované obrázky s plnými právy.

\section{Návrh a zdůvodnění technologií}

\subsection{HTML5, CSS3 a Vanilla JavaScript}

HTML5 byl vybrán pro bohatou sadu sémantických elementů zlepšujících strukturu, SEO a accessibility, nativní podporu multimédií a formulářů, a širokou podporu v moderních prohlížečích. CSS3 bez frameworků umožňuje moderní layoutové systémy (Grid, Flexbox) bez hacků, CSS variables pro snadnou správu designu, nativní transitions a animations a media queries pro responzivní design. Použití čistého CSS prokazuje fundamentální znalosti a je konceptuálně konzistentní s tématem projektu (web o rehabilitaci od AI závislosti ironicky používá čisté technologie). Vanilla JavaScript bez knihoven zajišťuje rychlost (85 řádků vs. desítky KB knihoven), využívá moderní nativní DOM API, usnadňuje maintenance a je součástí satirické zprávy projektu kritizující přílišnou závislost na nástrojích.

\subsection{Statický web a GitHub Pages}

Projekt je implementován jako statický web bez backendu pro jednoduchost (pouze HTML/CSS/JS soubory), bezpečnost (bez SQL injection, XSS rizik), rychlost a škálovatelnost. GitHub Pages byl vybrán jako hosting platforma pro bezplatný hosting s HTTPS, automatický deployment při push do repository, vysokou spolehlivost GitHub infrastruktury a kompletní verzování díky Git. Absence backendu zjednodušuje deployment a eliminuje potřebu správy serverů.

\begin{figure}[H]
\centering
\includegraphics[width=\textwidth]{screens/02/08-w3c-validation.png}
\caption{W3C HTML Validator -- výsledek validace Home page (index.html). Všechny stránky byly úspěšně prověřeny validátorem bez chyb.}
\end{figure}
