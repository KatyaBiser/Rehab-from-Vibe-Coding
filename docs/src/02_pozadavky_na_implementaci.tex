\chapter{Požadavky na implementaci}
\label{chap:pozadavky}

\section{Popis webových stránek}

Web se skládá z jedenácti HTML stránek: úvodní stránka, seznam terapeutů a devět detailních profilů. Každá stránka je navržena s důrazem na konzistenci designu, přístupnost a responzivní chování.

\subsection{Úvodní stránka (index.html)}

Úvodní stránka implementuje sémantické HTML elementy pro strukturu obsahu. Glosář odborných pojmů využívá \texttt{<dl>}, \texttt{<dt>}, \texttt{<dd>} elementy pro správnou sémantiku definition lists. Testimonials používají \texttt{<blockquote>} elementy s citacemi klientů. Footer je organizován pomocí CSS Grid layoutu s responzivním chováním -- transformace z jednoho sloupce na mobilech do čtyř sloupců na desktopech. Tabulka příznaků AI závislosti obsahuje osm řádků s využitím \texttt{<thead>}, \texttt{<tbody>}, \texttt{<th scope="col">} pro přístupnost.

\subsection{Seznam terapeutů (pages/therapists/list.html)}

Stránka implementuje CSS Grid layout s devíti položkami. Grid používá \texttt{display: grid} s responzivními sloupci: \texttt{grid-template-columns: 1fr} na mobilech, \texttt{repeat(2, 1fr)} na tabletech ($\geq$768px), \texttt{repeat(3, 1fr)} na desktopech ($\geq$1024px). Tři karty obsahují conditional badge element s třídou \texttt{.card\_\_badge} pro označení ``Lead Therapist''. Každá karta je \texttt{<article>} element s \texttt{<img>}, nadpisy a odkazem na detailní profil.

\subsection{Detailní profily terapeutů}

Každý terapeut má dedikovanou HTML stránku s breadcrumb navigací implementovanou pomocí \texttt{<nav>} s \texttt{<ol>} a \texttt{aria-current="page"} pro aktuální stránku. Recenze klientů používají \texttt{<blockquote>} elementy se sémantickou strukturou citací. Struktura dat zahrnuje: biografie (několik odstavců \texttt{<p>}), oblasti expertízy (neuspořádaný seznam \texttt{<ul>}), vybrané publikace (\texttt{<ul>}), vzdělání a certifikace (\texttt{<ul>}), terapeutický přístup (úvodní text + seznam) a call-to-action sekce s tlačítkem pro rezervaci.

\section{Chování a interaktivita stránek}

\subsection{Responzivní layout}

Web implementuje Mobile-first přístup s třemi breakpointy: Mobile (<768px) s výchozími styly a jedním sloupcem, Tablet ($\geq$768px) s dvousloupcovými layouty a horizontální navigací, Desktop ($\geq$1024px) s tří a čtyřsloupcovými layouty a maximální šířkou 1200px. Grid layout terapeutů se transformuje z jednoho sloupce přes dva na tři sloupce. Footer grid mění strukturu z jednoho na čtyři sloupce.

\subsection{Navigace}

Na mobilních zařízeních je menu defaultně skryté s hamburger tlačítkem. Navigace využívá řešení založené na checkbox elementu -- skrytý \texttt{<input type="checkbox"\allowbreak{} id="nav-toggle">} element je propojen s \texttt{<label for="nav-toggle">}, který funguje jako hamburger tlačítko. Při zaškrtnutí checkboxu CSS selektor \texttt{\#nav-toggle:checked\allowbreak{} \textasciitilde{}\allowbreak{} .nav\allowbreak{} .nav\_\_list} zobrazuje menu pomocí \texttt{display: flex}. Hamburger animace (transformace do křížku) je realizována pomocí CSS transforms na \texttt{.nav\_\_\allowbreak{}toggle-bar} elementy. Menu se automaticky zavírá při změně velikosti okna na tablet/desktop breakpoint díky media queries. Na větších obrazovkách je menu vždy viditelné jako horizontální seznam.

\subsection{Vyhledávání a interaktivní efekty}

Vyhledávací pole je implementováno jako dropdown aktivovaný tlačítkem s ikonou lupy. Stejně jako navigace používá CSS řešení s checkbox elementem -- \texttt{<input type="checkbox"\allowbreak{} id="search-toggle">} propojený s \texttt{<label>} ikonou. Při zaškrtnutí checkboxu CSS selektor \texttt{\#search-toggle:checked\allowbreak{} \textasciitilde{}\allowbreak{} .search-dropdown} zobrazuje dropdown pomocí \texttt{display: block} s CSS \texttt{@keyframes fadeIn} animací. Všechny interaktivní elementy mají hover efekty (karty s \texttt{transform:\allowbreak{} translateY(-5px)}, tlačítka s \texttt{transform:\allowbreak{} scale(1.02)}) a výrazné focus stavy s modrým obrysem (\texttt{:focus-visible}) pro klávesnicovou navigaci. Skip link je viditelný při focusu.

\section{HTML požadavky}

\subsection{Struktura dokumentu}

Všechny stránky používají HTML standard s \texttt{<!DOCTYPE html>}, \texttt{<html lang="en">}, meta tagy pro kódování UTF-8, viewport pro responzivitu, description, keywords a title specifický pro každou stránku. Favicon a apple-touch-icon jsou definovány. Všechny stránky linkují jeden centrální CSS soubor s relativním adresováním.

\subsection{Sémantické elementy}

Projekt využívá sémantické HTML elementy: \texttt{<header>} pro hlavičku s logem a navigací, \texttt{<nav>} s \texttt{aria-label}, \texttt{<main id="main-content">} jako target pro skip link, \texttt{<section>} pro tematické sekce, \texttt{<article>} pro karty a profily, \texttt{<aside>} pro testimonials, \texttt{<footer>} pro patičku. Dodržována je správná hierarchie nadpisů s jedním \texttt{<h1>} na stránku. Tabulka používá \texttt{<thead>}, \texttt{<tbody>}, \texttt{<th scope="col">}. Glosář využívá \texttt{<dl>}, \texttt{<dt>}, \texttt{<dd>}.

\subsection{Přístupnost (WCAG 2.0 Level AA)}

Web splňuje WCAG 2.0 Level AA požadavky. Skip link \texttt{<a href="\#main-content"} \texttt{class="skip-link">} je umístěn na začátku každé stránky. Použité jsou ARIA atributy (\texttt{aria-label}, \texttt{aria-expanded}, \texttt{aria-current="page"}). Screen reader texty používají třídu \texttt{sr-only}. Všechny obrázky mají smysluplné alt texty. Barevný kontrast je vyšší než 7:1 (tmavě šedá \texttt{\#212529} na bílém pozadí). Výrazné focus stavy zajišťují klávesnicovou navigaci.

\section{CSS požadavky}

\subsection{Architektura CSS}

Projekt používá 100\% vlastní CSS bez frameworků. Veškerý CSS je organizován v jediném souboru \texttt{assets/css/style.css}, který obsahuje všechny styly strukturované do logických sekcí s komentáři: CSS Variables, Reset \& Base Styles, Typography, Layout (Container, Header, Footer, Grid), Components (Buttons, Cards, Search, Hero, Tables, Breadcrumbs, Testimonials), Pages (Detail, Sections), Utilities (Accessibility, Helpers, Glossary) a Responsive Design. Celková velikost je 1249 řádků optimalizovaného kódu.

\subsection{CSS Custom Properties}

{\sloppy
Systém CSS Custom Properties je definován v \texttt{:root} sekci souboru \texttt{assets/css/}\allowbreak\texttt{style.css}.
\par}

Pokrývá barevnou paletu (\texttt{--primary-color}, \texttt{--secondary-color}, \texttt{--accent-color}, atd.) a typografii (\texttt{--font-primary}, \texttt{--font-heading}, velikosti písma).

Spacing (\texttt{--spacing-xs} až \texttt{--spacing-xxl} od 0.5rem do 4rem) je oddělená skupina proměnných. Layout obsahuje \texttt{--max-width-container} a \texttt{--border-radius}. Další proměnné: \texttt{--box-shadow}, \texttt{--transition-speed}. Proměnné zajišťují konzistenci a snadné globální úpravy designu.

\subsection{CSS Grid layout}

CSS Grid je použit pro layout seznamu terapeutů (\texttt{.grid--therapists}) a footer (\texttt{.footer\_\_grid}). Grid terapeutů obsahuje devět karet s responzivními sloupci: \texttt{grid-template-columns: 1fr} na mobilech, \texttt{repeat(2, 1fr)} na tabletech, \texttt{repeat(3, 1fr)} na desktopech. Gap mezi položkami je \texttt{var(--spacing-md)}. Footer grid používá čtyři sloupce na desktopech s adaptací na menší počet sloupců. Grid nabízí 2D kontrolu (řádky i sloupce) a jednodušší responzivní úpravy než Flexbox.

\subsection{Flexbox}

Flexbox je použit pro jednodimenzionální layouty: navigace (\texttt{.nav\_\_list} s flexbox pro horizontální/vertikální rozložení), header (\texttt{.header\_\_container} s \texttt{flex-direction: column/row} podle breakpointu), karty (\texttt{.card} s \texttt{flex-direction: column}) a tlačítka (\texttt{.btn} s \texttt{inline-flex; align-items: center; justify-content: center}).

\subsection{Responzivní design}

Mobile-first přístup používá základní styly pro nejmenší obrazovky. Media queries \texttt{@media (min-width: ...)} rozšiřují layout. Techniky zahrnují fluid typography pomocí relative units (rem, em), fluid layouts pomocí procent a CSS Grid/Flexbox. Responsive images s \texttt{max-width: 100\%; height: auto}, conditional visibility pomocí \texttt{display: none} a transformaci navigace z vertikální na horizontální.

\section{CSS interaktivita}

\subsection{Interaktivní CSS technika}

Mobilní navigace a vyhledávací dropdown jsou realizovány pomocí checkbox elementů a CSS selektorů -- elegantního řešení, které plně odpovídá filozofii projektu -- demonstraci fundamentálních webových dovedností.

Tato technika využívá kombinaci skrytého \texttt{<input type="checkbox">} elementu, \texttt{<label>} elementu a CSS selektorů pro vytvoření interaktivních komponent. Hamburger menu používá \texttt{\#nav-toggle:checked\allowbreak{} \textasciitilde{}\allowbreak{} .nav\allowbreak{} .nav\_\_list} selektor pro zobrazení navigace. Vyhledávací dropdown funguje analogicky s \texttt{\#search-toggle:checked\allowbreak{} \textasciitilde{}\allowbreak{} .search-dropdown}. Animace hamburger ikony (transformace do křížku) je realizována pomocí CSS \texttt{transform} vlastností na \texttt{.nav\_\_\allowbreak{}toggle-bar} elementy. ARIA atributy (\texttt{aria-label}, \texttt{aria-current}) jsou staticky definovány v HTML a zajišťují přístupnost pro screen readery. Všechny hover efekty, focus stavy a transitions jsou implementovány v CSS pomocí pseudo-tříd \texttt{:hover}, \texttt{:focus}, \texttt{:focus-visible}.
\subsection{Implementace}

Navigace používá skrytý \texttt{<input type="checkbox"\allowbreak{} id="nav-toggle">} element propojený s \texttt{<label for="nav-toggle">}, který funguje jako hamburger tlačítko. CSS selektor \texttt{\#nav-toggle:checked\allowbreak{} \textasciitilde{}\allowbreak{} .nav\allowbreak{} .nav\_\_list} zobrazí menu při zaškrtnutí checkboxu. Vyhledávací dropdown funguje analogicky s \texttt{id="search-toggle"} a CSS animacemi pomocí \texttt{@keyframes fadeIn}. Toto řešení zajišťuje interaktivitu bez externí závislosti, s plnou podporou accessibility (ARIA atributy \texttt{aria-label} na label elementech) a rychlým výkonem díky nativním CSS selektorům.

\section{Návrh a zdůvodnění technologií}

\subsection{HTML a CSS}

HTML byl vybrán pro bohatou sadu sémantických elementů zlepšujících strukturu, SEO a accessibility, nativní podporu multimédií a formulářů, a širokou podporu v moderních prohlížečích. CSS bez frameworků umožňuje moderní layoutové systémy (Grid, Flexbox) s čistou implementací, CSS variables pro snadnou správu designu, nativní transitions a animations, media queries pro responzivní design a pokročilé techniky jako checkbox-based interaktivitu pomocí \texttt{:checked} selektorů. Použití čistých technologií prokazuje fundamentální znalosti a je konceptuálně konzistentní s tématem projektu (web o rehabilitaci od AI závislosti demonstruje základní webové dovednosti). Tento přístup přináší rychlejší načítání (1271 řádků CSS bez externích závislostí), vynikající výkon a plnou kontrolu nad chováním UI elementů.

\subsection{Statický web a GitHub Pages}

Projekt je implementován jako statický web bez backendu pro jednoduchost (pouze HTML/CSS soubory), bezpečnost (bez SQL injection, XSS rizik), rychlost a škálovatelnost. GitHub Pages byl vybrán jako hosting platforma pro bezplatný hosting s HTTPS, automatický deployment při push do repository, vysokou spolehlivost GitHub infrastruktury a kompletní verzování díky Git. Absence backendu zjednodušuje deployment a eliminuje potřebu správy serverů.
