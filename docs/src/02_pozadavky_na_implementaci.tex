\chapter{Požadavky na implementaci}
\label{chap:pozadavky}

\section{Popis webových stránek}

\begin{itemize}
  \item 12 HTML stránek: homepage, stránka s výsledky vyhledávání, seznam terapeutů, 9 detailních profilů.
  \item Důraz na konzistentní design, přístupnost a responzivní chování.
\end{itemize}

\subsection{Úvodní stránka (index.html)}

\begin{itemize}
  \item Sémantické prvky pro hlavní obsah; glosář přes \texttt{<dl>}, \texttt{<dt>}, \texttt{<dd>}.
  \item Testimonials v \texttt{<blockquote>} s citacemi klientů.
  \item Footer v CSS Grid: 1 sloupec (mobile), 2 sloupce (tablet), 4 sloupce (desktop).
  \item Tabulka příznaků AI závislosti s \texttt{<thead>}, \texttt{<tbody>}, \texttt{<th scope="col">}.
\end{itemize}

\subsection{Seznam terapeutů (pages/therapists/list.html)}

\begin{itemize}
  \item CSS Grid s 9 kartami: 1 sloupec (mobile), 2 sloupce (tablet), 4 sloupce (desktop).
  \item Badge \texttt{.therapist-card\_\_badge--premium} pro označení Lead Therapist.
  \item Každá karta je \texttt{<article>} s \texttt{<img>}, nadpisy a odkazem na detail.
\end{itemize}

\subsection{Detailní profily terapeutů}

\begin{itemize}
  \item Breadcrumbs v \texttt{<nav>} + \texttt{<ol>} s \texttt{aria-current="page"}.
  \item Recenze klientů v \texttt{<blockquote>}.
  \item Obsah: bio (\texttt{<p>}), expertíza (\texttt{<ul>}), publikace (\texttt{<ul>}), vzdělání (\texttt{<ul>}), terapeutický přístup (text + seznam), CTA na rezervaci.
\end{itemize}

\section{Chování a interaktivita stránek}

\subsection{Responzivní layout}

\begin{itemize}
  \item Mobile-first: výchozí styly a 1 sloupec pro mobily (<768px).
  \item Tablet ($\geq$768px): 2 sloupce, horizontální navigace.
  \item Desktop ($\geq$1024px): 3--4 sloupce, max šířka 1200px.
  \item Grid terapeutů: 1 \textrightarrow{} 2 \textrightarrow{} 4 sloupce; footer grid: 1 \textrightarrow{} 2 \textrightarrow{} 4 sloupce.
\end{itemize}

\subsection{Navigace}

\begin{itemize}
  \item Mobile: menu skryté, otevření přes hamburger (\texttt{<input type="checkbox" id="nav-toggle">} + \texttt{<label>}).
  \item Zobrazení menu: selektor \texttt{\#nav-toggle:checked \textasciitilde{} .nav .nav\_\_list} \textrightarrow{} \texttt{display: flex}.
  \item Animace hamburgeru do křížku přes CSS transform na \texttt{.nav\_\_toggle-bar}.
  \item Na tabletu/desktopu je menu vždy viditelné, při resize se mobilní stav zavře media query.
\end{itemize}

\subsection{Vyhledávání a interaktivní efekty}

\begin{itemize}
  \item Vyhledávání jako dropdown: \texttt{<input type="checkbox" id="search-toggle">} + \texttt{<label>} s ikonou lupy.
  \item Selektor \texttt{\#search-toggle:checked \textasciitilde{} .search-dropdown} \textrightarrow{} \texttt{display: block} + animace \texttt{@keyframes fadeIn}.
  \item Hover efekty: karty \texttt{transform: translateY(-2px)}, tlačítka \texttt{translateY(-1px)}.
  \item Přístupnost: výrazné \texttt{:focus-visible}, viditelný skip link při focusu.
\end{itemize}

\section{HTML požadavky}

\subsection{Struktura dokumentu}

\begin{itemize}
  \item \texttt{<!DOCTYPE html>}, \texttt{<html lang="en">}, meta UTF-8 + viewport na všech stránkách.
  \item Vlastní description + title per stránka, meta keywords na homepage.
  \item Jeden centrální CSS soubor, všude relativní adresy.
\end{itemize}

\subsection{Sémantické elementy}

\begin{itemize}
  \item \texttt{<header>}, \texttt{<nav>} (\texttt{aria-label}), \texttt{<main id="main-content">}, \texttt{<section>}, \texttt{<article>}, \texttt{<aside>}, \texttt{<footer>}.
  \item Nadpisová hierarchie s jedním \texttt{<h1>} na stránku; \texttt{<h2>}/\texttt{<h3>} pro podnadpisy.
  \item Tabulka: \texttt{<thead>}, \texttt{<tbody>}, \texttt{<th scope="col">}.
  \item Glosář: \texttt{<dl>}, \texttt{<dt>}, \texttt{<dd>}.
\end{itemize}

\subsection{Přístupnost (WCAG 2.0 Level AA)}

\begin{itemize}
  \item Skip link na začátku každé stránky: \texttt{<a href="\#main-content" class="skip-link">}.
  \item ARIA atributy: \texttt{aria-label}, \texttt{aria-current="page"}; screen reader texty přes třídu \texttt{sr-only}.
  \item Smysluplné alt texty, kontrast > 7:1 (\texttt{\#212529} na \texttt{\#ffffff}).
  \item Viditelné \texttt{:focus-visible}, klávesnicová navigace ověřena.
\end{itemize}

\section{CSS požadavky}

\subsection{Architektura CSS}

\begin{itemize}
  \item 100\% vlastní CSS bez frameworků v jednom souboru \texttt{style.css}.
  \item Struktura: CSS Variables, Reset \& Base, Typography, Layout (Container, Header, Footer, Grid), Components (Buttons, Cards, Search, Hero, Tables, Breadcrumbs, Testimonials), Pages, Utilities, Responsive Design.
  \item Velikost: 1249 řádků optimalizovaného kódu.
\end{itemize}

\subsection{CSS Custom Properties}

\begin{itemize}
  \item Definováno v \texttt{:root} v \texttt{style.css}.
  \item Barvy: \texttt{--primary-color}, \texttt{--secondary-color}, \texttt{--accent-color}, atd.
  \item Typografie: \texttt{--font-primary}, \texttt{--font-heading}, velikosti písma.
  \item Spacing: \texttt{--spacing-xs} až \texttt{--spacing-xxl} (0.5rem--4rem).
  \item Layout: \texttt{--max-width-container}, \texttt{--border-radius}, \texttt{--box-shadow}, \texttt{--transition-speed}.
\end{itemize}

\subsection{CSS Grid layout}

\begin{itemize}
  \item Grid pro \texttt{.grid--therapists} a \texttt{.footer\_\_grid}.
  \item Terapeuti: 1/2/4 sloupce dle breakpointu, \texttt{gap: var(--spacing-md)}.
  \item Footer: až 4 sloupce na desktopu s adaptací na menší displeje.
  \item 2D řízení layoutu, vhodnější než flex pro rovnoměrné rozmístění a spanování.
\end{itemize}

\subsection{Flexbox}

\begin{itemize}
  \item Navigace (\texttt{.nav\_\_list}) pro horizontální/vertikální rozložení.
  \item Header (\texttt{.header\_\_container}) s přepínáním \texttt{flex-direction} podle breakpointu.
  \item Karty (\texttt{.card}) s vertikálním stackem; tlačítka \texttt{.btn} jako \texttt{inline-flex}/\texttt{inline-block}.
\end{itemize}

\subsection{Responzivní design}

\begin{itemize}
  \item Mobile-first, rozšíření layoutu pomocí \texttt{@media (min-width: ...)}.
  \item Fluid typography (rem, em) a fluid layouty (procenta, Grid/Flex).
  \item Responsive images (\texttt{max-width: 100\%; height: auto}), podmíněná viditelnost \texttt{display: none}.
  \item Navigace se mění z vertikální (mobile) na horizontální (tablet/desktop).
\end{itemize}

\section{CSS interaktivita}

\subsection{Interaktivní CSS technika}

Mobilní navigace a vyhledávací dropdown jsou realizovány pomocí checkbox elementů a CSS selektorů -- elegantního řešení, které plně odpovídá filozofii projektu: demonstrace fundamentálních webových dovedností.
\begin{itemize}
  \item Kombinace skrytého \texttt{<input type="checkbox">}, \texttt{<label>} a selektorů \texttt{:checked} pro togglování.
  \item Hamburger: \texttt{\#nav-toggle:checked \textasciitilde{} .nav .nav\_\_list} zobrazí menu; animace ikony přes \texttt{transform} na \texttt{.nav\_\_toggle-bar}.
  \item Vyhledávání: \texttt{\#search-toggle:checked \textasciitilde{} .search-dropdown} + animace \texttt{@keyframes fadeIn}.
  \item ARIA: \texttt{aria-label}, \texttt{aria-current} staticky v HTML; hover/focus/transition stavy v CSS.
\end{itemize}
\subsection{Implementace}

\begin{itemize}
  \item Navigace: \texttt{<input type="checkbox" id="nav-toggle">} + \texttt{<label for="nav-toggle">}; zobrazení menu přes \texttt{\#nav-toggle:checked \textasciitilde{} .nav .nav\_\_list}.
  \item Vyhledávání: \texttt{id="search-toggle"} + \texttt{@keyframes fadeIn} pro dropdown.
  \item Žádné externí JS knihovny; interaktivita čistě CSS + ARIA atributy pro přístupnost.
\end{itemize}

\section{Návrh a zdůvodnění technologií}

\subsection{HTML a CSS}

\begin{itemize}
  \item HTML: sémantické elementy, SEO a přístupnost out-of-the-box, podpora multimédií a formulářů.
  \item CSS bez frameworků: Grid/Flex, custom properties, transitions/animations, media queries, \texttt{:checked} interakce.
  \item Konzistence s tématem: demonstrace fundamentálních dovedností místo závislosti na nástrojích.
  \item Výkon: žádné externí závislosti, plná kontrola nad UI, 1271 řádků optimalizovaného CSS.
\end{itemize}

\subsection{Statický web a GitHub Pages}

\begin{itemize}
  \item Statický web bez backendu: jednoduchost, žádné SQL/XSS riziko ze serveru, rychlý load.
  \item Hosting: GitHub Pages s HTTPS, automatický deploy při push, spolehlivá infrastruktura a verzování v Git.
  \item Odpadá správa serverů, deployment je plně automatizovaný.
\end{itemize}

\subsection{Nasazení, údržba a provoz}
\begin{itemize}
  \item \textbf{Hosting} (realizováno): GitHub Pages (statický obsah, automatický HTTPS, verzování přes Git).
  \item \textbf{Deploy pipeline} (realizováno): GitHub Actions workflow \texttt{deploy.yml} (\texttt{actions/configure-pages}, \texttt{upload-pages-artifact}, \texttt{deploy-pages}); trigger na push do \texttt{main}.
  \item \textbf{Další kroky, pokud budeme projekt rozvíjet}: lehká privacy-friendly analytika (Plausible/GA4) jako jeden embed skript s eventy na CTA a vyhledávání; statická sitemap a og meta (\texttt{og:title}/\texttt{og:description}/\texttt{og:image}) pro sdílení; pravidelná údržba (měsíční kontrola actions verzí, vizuální smoke test po deploy); kvartální UX/accessibility review (klávesnice, kontrast, Lighthouse/W3C validator).
\end{itemize}
