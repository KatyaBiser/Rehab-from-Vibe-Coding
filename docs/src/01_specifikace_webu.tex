\chapter{Úvod}

\section{Představení řešeného webu}

\subsection{Název projektu}
\textbf{Rehab from Vibe Coding}

\subsection{Popis zařízení}
Jedná se o fiktivní rehabilitační centrum specializující se na léčbu vývojářů
závislých na AI asistovaném programování. Centrum nabízí terapeutické programy
zaměřené na obnovu fundamentálních programovacích dovedností, sémantického HTML,
CSS bez frameworků, přístupnosti a bezpečnosti kódu.

\subsection{Koncept webu}
Web satiricky reflektuje současný trend nadměrného spoléhání se na AI nástroje
při vývoji softwaru. Nabízí humorný pohled na problémy vývojářů, kteří ztratili
schopnost psát kvalitní kód bez asistence AI.

\section{Struktura a popis stránek}

\subsection{Úvodní stránka (index.html)}
\begin{itemize}
    \item Hero sekce s výrazným nadpisem a popisem služeb
    \item Sekce ``About Our Mission'' s představením poslání centra
    \item Tabulka příznaků AI závislosti s doporučenými programy
    \item Glosář odborných pojmů (Vibe Coding, Div Soup, atd.)
    \item Sekce s přehledem léčebných programů
    \item Call-to-action sekce pro rezervaci konzultace
    \item Testimonials (úspěšné příběhy klientů)
    \item Plnohodnotná patička s kontakty a informacemi o týmu
\end{itemize}

\subsection{Seznam terapeutů (pages/therapists/list.html)}
\begin{itemize}
    \item Přehled 8 terapeutů v grid layoutu
    \item Každý terapeut má fotku, jméno, specializaci a stručný popis
    \item Odkazy na detailní profily jednotlivých terapeutů
    \item Vyhledávací pole pro konzistenci napříč stránkami
\end{itemize}

\subsection{Detailní profily terapeutů (8 samostatných stránek)}
\begin{enumerate}
    \item \textbf{emma-html.html} -- Dr. Emma HTML (Semantic Markup Specialist)
    \item \textbf{marcus-grid.html} -- Prof. Marcus Grid (Layout Architecture Expert)
    \item \textbf{aria-accessible.html} -- Dr. Aria Accessible (Web Accessibility Advocate)
    \item \textbf{flex-box.html} -- Dr. Flex Box (Component Layout Specialist)
    \item \textbf{val-root.html} -- Specialist Val Root (CSS Variables \& Architecture)
    \item \textbf{sam-security.html} -- Dr. Sam Security (Code Security Consultant)
    \item \textbf{petra-perf.html} -- Prof. Petra Perf (Performance Optimization)
    \item \textbf{min-code.html} -- Dr. Min Code (Mindful Coding Coach)
\end{enumerate}

Každý profil obsahuje:
\begin{itemize}
    \item Portrét a základní informace
    \item Detailní biografii
    \item Seznam specializací a odborností
    \item Publikace a články
    \item Vzdělání a certifikace
    \item Terapeutický přístup
    \item Recenze od klientů
    \item Call-to-action pro rezervaci sezení
\end{itemize}

\subsection{Vyhledávání}
Vyhledávací pole je přítomno na všech stránkách, připraveno pro budoucí
implementaci funkčního vyhledávání.

\section{Cíl webu a cílové skupiny}

Web představuje rehabilitační centrum a jeho služby, informuje o problematice AI
závislosti ve vývoji a nabízí prezentaci 8 odborníků. Cílové skupiny zahrnují junior
vývojáře, frontend developery s nedostatečnými znalostmi HTML/CSS, týmové lídry a
studenty informatiky. Web je plně responzivní s vysokou úrovní přístupnosti (WCAG 2.0 AA).

\section{Požadavky na implementaci}

\subsection{HTML požadavky}
\begin{itemize}
    \item HTML5 s korektní strukturou
    \item Sémantické elementy (article, section, header, footer, aside, nav, main)
    \item Správné použití nadpisů h1, h2, h3
    \item Tabulka pro strukturovaná data
    \item Formulářové elementy pro vyhledávání
    \item Relativní adresování všech odkazů a zdrojů
    \item Meta tagy pro SEO a responzivitu
\end{itemize}

\subsection{CSS požadavky}
\begin{itemize}
    \item Vlastní CSS bez použití frameworků
    \item CSS Custom Properties (proměnné) pro barvy, spacing, typography
    \item CSS Grid layout s minimálně 8 elementy
    \item Flexbox pro layouty komponent
    \item Responzivní design s Mobile-first přístupem
    \item Media queries pro tablet (768px+) a desktop (1024px+)
\end{itemize}

\subsection{Přístupnost}
\begin{itemize}
    \item WCAG 2.0 Level AA compliance
    \item Skip links pro klávesnicovou navigaci
    \item ARIA labels a roles kde potřebné
    \item Sémantické HTML pro screen readery
    \item Dostatečný kontrast barev
    \item Focus states pro interaktivní elementy
\end{itemize}

\subsection{Chování stránek}
\begin{itemize}
    \item Responzivní layout se přizpůsobí šířce obrazovky
    \item Grid layout se mění: 1 sloupec (mobil) $\rightarrow$ 2 sloupce (tablet)
          $\rightarrow$ 3-4 sloupce (desktop)
    \item Navigace se transformuje z vertikálního menu na horizontální
    \item Hover efekty na kartách terapeutů a tlačítkách
    \item Smooth transitions pro interaktivní elementy
    \item Print styles pro tisk dokumentů
\end{itemize}

\section{Návrh technologií}

Projekt využívá HTML5, CSS3 (Grid, Flexbox, Custom Properties) a responzivní Mobile-first přístup. Zaměření na čisté HTML/CSS řešení bez frameworků demonstruje fundamentální dovednosti. Web je nasazen na GitHub Pages (\url{https://katyabiser.github.io/Rehab-from-Vibe-Coding/}).

% TODO: Přidat screenshot úvodní stránky (index.html) - hero sekce a about
% TODO: Přidat screenshot seznamu terapeutů (therapists/list.html) - grid layout
% TODO: Přidat screenshot detailu terapeuta - např. emma-html.html
% TODO: Přidat screenshot responzivního designu - mobile vs desktop srovnání
