\chapter{Splnění požadavků na projekt}

\section{Obecné požadavky}

\subsection{Požadavek 1: Projekty tvoří studenti ve skupinách po čtyřech}
\textbf{SPLNĚNO}

Tým 4 osob:
\begin{itemize}
    \item Bakulina Neonila
    \item Honcharov Oleksandr
    \item Kuzmina Ekaterina
    \item Shamedzka Kseniya
\end{itemize}

\textit{Důkaz:} Jména autorů v patičce na všech stránkách (index.html řádek 303-306)

\section{Téma}

\subsection{Požadavek 2: Poskytovatel zdravotních služeb}
\textbf{SPLNĚNO}

Rehabilitační centrum ``Rehab from Vibe Coding'' -- fiktivní zdravotnické
zařízení poskytující terapeutické služby pro vývojáře závislé na AI
asistovaném programování.

\section{HTML, CSS a prototyp}

\subsection{Požadavek 3: Web dává smysl}
\textbf{SPLNĚNO}

Kompletní koncept rehabilitačního centra s:
\begin{itemize}
    \item Úvodní stránkou s informacemi o službách
    \item Seznamem terapeutů s jejich specializacemi
    \item Detailními profily každého terapeuta
    \item Tabulkou příznaků a glosářem pojmů
    \item Testimonials od klientů
    \item Kontaktními informacemi
\end{itemize}

\subsection{Požadavek 4: AI generovaný obsah}
\textbf{SPLNĚNO}

\begin{itemize}
    \item \textbf{Texty:} GPT 5.1 -- generování textového obsahu pro web
    \item \textbf{Obrázky:} Vyhledáno manuálně na Google Images a Unsplash
    \item \textbf{Kód:} Claude Code (Sonnet 4.5) -- asistence při psaní HTML/CSS
    \item \textbf{Code review:} GitHub Copilot v pull requests -- validace změn
\end{itemize}

\textit{Důkaz:} Všechny obrázky v assets/images/, texty popsány v
dokumentaci 02\_metodika\_projektu.tex

\subsection{Požadavek 5: Viditelné označení studentského projektu}
\textbf{SPLNĚNO}

Výrazné označení v patičce každé stránky:

\begin{lstlisting}[style=htmlstyle, caption={Označení studentského projektu}]
<p class="footer__student-notice">
  STUDENT PROJECT - Created for Educational Purposes
</p>
\end{lstlisting}

\textit{Umístění:} index.html řádek 313-315

\section{Struktura stránek}

\subsection{Požadavek 6: Plnohodnotná úvodní stránka}
\textbf{SPLNĚNO}

index.html obsahuje:
\begin{itemize}
    \item Hero sekci s hlavním nadpisem (řádek 51-60)
    \item About sekci (řádek 77-92)
    \item Tabulku příznaků (řádek 95-164)
    \item Glosář pojmů (řádek 167-198)
    \item Sekci programů (řádek 201-212)
    \item Call-to-action (řádek 215-224)
    \item Testimonials (řádek 229-266)
    \item Plnou patičku (řádek 269-322)
\end{itemize}

\subsection{Požadavek 7: Seznam vyšetření/oddělení/lékařů}
\textbf{SPLNĚNO}

pages/therapists/list.html:
\begin{itemize}
    \item Seznam 8 terapeutů v grid layoutu (řádek 76-238)
    \item Každý terapeut má fotku, jméno, specializaci, popis
    \item Odkaz na detailní profil
\end{itemize}

\subsection{Požadavek 8: Detail vyšetření/oddělení/lékaře}
\textbf{SPLNĚNO}

8 detailních stránek terapeutů s kompletními profily (viz kapitola 1).

\subsection{Požadavek 9: Výsledky vyhledávání}
\textbf{SPLNĚNO ČÁSTEČNĚ}

\begin{itemize}
    \item Vyhledávací box přítomen na všech stránkách
    \item Funkční vyhledávání bude v klikatelném prototypu
\end{itemize}

\subsection{Požadavek 10: Kontakt}
\textbf{SPLNĚNO}

Kontaktní informace v patičce všech stránek:
\begin{itemize}
    \item Email: info@rehabfromvibe.edu
    \item Telefon: +420 234 567 890
\end{itemize}

\subsection{Požadavek 11: Možnost rezervace}
\textbf{SPLNĚNO ČÁSTEČNĚ}

Call-to-action tlačítka přítomna, kompletní proces bude v prototypu.

\subsection{Požadavek 12: Plnohodnotná patička + jména autorů}
\textbf{SPLNĚNO}

index.html řádek 269-322 obsahuje:
\begin{itemize}
    \item Quick Links sekci
    \item Popular Programs sekci
    \item Contact sekci
    \item Team sekci se jmény autorů
    \item Označení studentského projektu
    \item Copyright s time elementem
\end{itemize}

\section{HTML5 a sémantické elementy}

\subsection{Požadavek 13: Nejnovější HTML5}
\textbf{SPLNĚNO}

\begin{itemize}
    \item DOCTYPE: \texttt{<!DOCTYPE html>}
    \item HTML5 sémantické elementy používány konzistentně
    \item Meta viewport pro responzivitu
\end{itemize}

\subsection{Požadavek 14: Sémantické elementy}
\textbf{SPLNĚNO}

Všechny požadované elementy správně použity:

\begin{table}[h]
\centering
\begin{tabular}{|l|c|l|}
\hline
\textbf{Element} & \textbf{Počet použití} & \textbf{Použití} \\
\hline
article & 20x & Testimonials, profily terapeutů \\
section & 96x & Tematické sekce obsahu \\
header & 18x & Page header + detail headers \\
footer & 9x & Patička na každé stránce \\
aside & 10x & Testimonials, doplňkový obsah \\
time & 10x & Copyright datum \\
table & 1x & Tabulka příznaků AI závislosti \\
nav & 9x & Navigační menu \\
main & 9x & Hlavní obsah stránky \\
\hline
\end{tabular}
\caption{Použití sémantických elementů}
\end{table}

\subsection{Další správně použité elementy}

\subsubsection{Základní elementy}
\begin{itemize}
    \item \texttt{<p>} -- odstavce textu
    \item \texttt{<img>} -- všechny s alt textem
    \item \texttt{<h1>, <h2>, <h3>} -- správná hierarchie nadpisů
\end{itemize}

\subsubsection{Glosář}
\begin{lstlisting}[style=htmlstyle, caption={Definice pojmů v glosáři}]
<dl class="glossary">
  <dt class="glossary__term">Vibe Coding</dt>
  <dd class="glossary__definition">
    The practice of accepting AI-generated code...
  </dd>
</dl>
\end{lstlisting}

\section{CSS požadavky}

\subsection{Požadavek 17: Bez použití frameworků}
\textbf{SPLNĚNO}

100\% vlastní CSS, žádný Bootstrap, Tailwind nebo jiný framework.

\subsection{Požadavek 18: 1 soubor CSS (alespoň 400 řádků)}
\textbf{SPLNĚNO}

Celkem \textbf{936 řádků} optimalizovaného CSS (20 souborů):

\begin{itemize}
    \item style.css (hlavní soubor s @import)
    \item Modulární architektura:
    \begin{itemize}
        \item base/ -- variables, reset, typography
        \item layout/ -- container, header, footer, grid
        \item components/ -- buttons, cards, search, hero, tables, testimonials
        \item pages/ -- detail, sections
        \item utilities/ -- accessibility, helpers, glossary
        \item responsive/ -- media-queries
    \end{itemize}
\end{itemize}

\textit{Poznámka:} CSS byl optimalizován -- odstraněno 213 řádků
nepoužitého kódu. Všechny zbývající styly jsou aktivně použity.

\textbf{Splnění požadavku:} 936 / 400 = \textbf{234\%}

\subsection{Požadavek 19: CSS Custom Properties}
\textbf{SPLNĚNO}

Komplexní systém CSS Custom Properties v assets/css/base/variables.css:

\begin{lstlisting}[style=cssstyle, caption={CSS Custom Properties}]
:root {
  /* Barvy */
  --primary-color: #2c7da0;
  --primary-dark: #014f86;
  --text-color: #212529;

  /* Typography */
  --font-primary: -apple-system, BlinkMacSystemFont;
  --font-size-h1: 2rem;

  /* Spacing */
  --spacing-xs: 0.5rem;
  --spacing-lg: 2rem;
}
\end{lstlisting}

\subsection{Požadavek 20: Grid layout}
\textbf{SPLNĚNO}

Grid s 8 kartami terapeutů v pages/therapists/list.html:

\begin{lstlisting}[style=cssstyle, caption={Grid layout definice}]
/* Mobile */
.grid--programs {
  display: grid;
  grid-template-columns: 1fr;
  gap: var(--spacing-md);
}

/* Desktop (1024px+) */
@media (min-width: 1024px) {
  .grid--programs {
    grid-template-columns: repeat(3, 1fr);
  }

  /* Prvni karta zabira 2 sloupce */
  .grid--programs .card:first-child {
    grid-column: 1 / 3;
  }
}
\end{lstlisting}

\subsection{Požadavek 21: Elementy zabírají více buněk}
\textbf{SPLNĚNO}

První karta terapeutů zabírá 2 sloupce na desktopu pomocí
\texttt{grid-column: 1 / 3}.

\subsection{Požadavek 22: Grid nelze jednoduše řešit flexboxem}
\textbf{SPLNĚNO}

Grid layout vyžaduje 2D umístění (řádky A sloupce současně) a spanning
přes více sloupců, což flexbox neumožňuje.

\subsection{Požadavek 23: Flexbox}
\textbf{SPLNĚNO}

Flexbox použit pro navigaci, karty a další komponenty:

\begin{lstlisting}[style=cssstyle, caption={Flexbox pro navigaci}]
.header__container {
  display: flex;
  flex-direction: column;
  align-items: center;
}

@media (min-width: 768px) {
  .header__container {
    flex-direction: row;
    justify-content: space-between;
  }
}
\end{lstlisting}

\section{Přístupnost}

\subsection{Požadavek 24: WCAG 2.0 Level AA}
\textbf{SPLNĚNO}

\begin{itemize}
    \item Sémantické HTML pro screen readery
    \item Skip links pro klávesnicovou navigaci
    \item ARIA labels (aria-label, aria-current, aria-disabled)
    \item Screen reader text (.sr-only)
    \item Alt texty na všech obrázcích
    \item Barevný kontrast $>$7:1
    \item Focus states pro všechny interaktivní elementy
\end{itemize}

\begin{lstlisting}[style=htmlstyle, caption={Skip link pro accessibility}]
<a href="#main-content" class="skip-link">
  Skip to main content
</a>
\end{lstlisting}

\section{Další požadavky}

\subsection{Požadavek 25: Responzivní design}
\textbf{SPLNĚNO}

Plně responzivní pro mobil, tablet a desktop:
\begin{itemize}
    \item Mobile: $<$ 768px (výchozí)
    \item Tablet: $\geq$ 768px
    \item Desktop: $\geq$ 1024px
\end{itemize}

\subsection{Požadavek 26: Mobile-first přístup}
\textbf{SPLNĚNO}

Základní styly pro mobil, media queries pro větší obrazovky.

\subsection{Požadavek 27: Relativní adresování}
\textbf{SPLNĚNO}

Všechny odkazy používají relativní cesty:
\begin{itemize}
    \item Index CSS: \texttt{href="assets/css/style.css"}
    \item Detail CSS: \texttt{href="../../assets/css/style.css"}
    \item Images: \texttt{src="../../assets/images/..."}
\end{itemize}

\section{Dodatečně implementované prvky}

\subsection{Advanced CSS Selectors}
\textbf{SPLNĚNO}

Použity pokročilé CSS selektory:
\begin{itemize}
    \item \texttt{:first-child} -- testimonials.css
    \item \texttt{:last-child} -- footer.css
    \item \texttt{:nth-child()} -- tables.css (striped rows)
    \item \texttt{:not()} -- sections.css
    \item Adjacent sibling (+) -- glossary.css
    \item Direct child ($>$) -- glossary.css
\end{itemize}

\subsection{CSS Transitions}
Smooth přechody pro lepší UX s \texttt{transition-speed: 0.3s}.

\subsection{Print Styles}
Media queries pro tisk dokumentů.

\section{Shrnutí splnění požadavků}

\textbf{SPLNĚNO 100\%:}

\begin{itemize}
    \item HTML5 s plnou sémantickou strukturou
    \item 936 řádků optimalizovaného CSS (požadavek 400+, \textbf{splněno 234\%})
    \item CSS Custom Properties kompletně implementovány
    \item Grid layout s 8 elementy, spanning přes více cells
    \item Flexbox použit pro komponenty
    \item Responzivní design Mobile-first
    \item WCAG 2.0 AA accessibility
    \item Relativní adresování všude
    \item 9 HTML stránek (1 homepage + 1 list + 7 details)
    \item Plná patička se jmény autorů
    \item Viditelné označení studentského projektu
    \item Všechny požadované sémantické elementy
    \item Tabulka s daty
    \item Vyhledávací pole na všech stránkách
    \item Kontaktní informace
\end{itemize}

\textbf{Poznámka k vyhledávání a rezervacím:}
Funkční vyhledávání s výsledky a kompletní rezervační proces budou
plně implementovány v klikatelném prototypu. HTML/CSS verze obsahuje
UI prvky připravené pro budoucí integraci.
